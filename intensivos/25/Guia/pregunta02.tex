\section{Una de diccionarios de diccionarios}

Pearson Hardman, una firma muy prestigiosa de abogados, requiere ayuda con el manejo de datos de sus empleados. Teniendo el diccionario \texttt{abogados} cuyas llaves son nombres y sus valores otros diccionarios. Estos diccionarios asociados a cada abogado contienen datos de:
\begin{itemize}
    \item Los juicios realizados por mes, bajo la llave \texttt{'juicios'} que asocia una lista de tuplas con la forma \texttt{(mes,cantidad)}
    \item El sueldo que gana por hora el abogado, bajo la llave \texttt{'sueldo'} que asocia un entero
    \item Las empresas que ha defendido el abogado bajo la llave \texttt{'empresas'} que asocia una lista de strings
\end{itemize}

\begin{lstlisting}[style=consola]
abogados={
    'mike':{'juicios':[('julio',3),('agosto',1),('octubre',4)],
            'sueldo':100,
            'empresas':['google','samsung','ciac'] },
    'rachel':{'juicios':[('enero',3),('marzo',4),('julio',8)],
              'sueldo':70,
              'empresas':['google','codelco']},
    'harvey':{'juicios':[('enero',5),('febrero',12),('julio',24)],
              'sueldo':1000,
              'empresas':['mesa verde','samsung']}
    }
\end{lstlisting}

Se le pide a usted crear las siguientes funciones
\begin{itemize}
    \item[a.] \texttt{juicios\_por\_mes(abogados)} que reciba el diccionario \texttt{abogados} y retorne un diccionario que asocie el mes con la cantidad total de juicios realizados.
    \begin{lstlisting}[style=consola]
>>> [*juicios_por_mes(abogados)*]
{'julio': 35, 'marzo': 4, 'agosto': 1, 'enero': 8, 'febrero': 12, 
'octubre': 4}
    \end{lstlisting}
    
    \item[b.] \texttt{total\_juicios(abogado)} que reciba el nombre de un abogado y retorne un entero con la cantidad total de juicios en los que ha estado. Asuma que el diccionario \texttt{abogados} es una variable global del programa.
    \begin{lstlisting}[style=consola]
>>> [*total_juicios('harvey')*]
41
    \end{lstlisting}
    
    \item[c.] \texttt{quien\_trabajo(empresa)} que reciba un string con una empresa y retorne una lista con todos los nombres de los abogados que trabajaron en dicha empresa.
    \begin{lstlisting}[style=consola]
[*>>> quien_trabajo('google')*]
['mike', 'rachel']
    \end{lstlisting}
%    \item[d.] \texttt{se\_conocen(abogado\_1,abogado\_2)} que retorne un valor booleano si es que ambos abogados se conocen. Considere que se conocen si es que han trabajado en la misma empresa.
%    \begin{lstlisting}[style=consola]
%>>> [*se_conocen('mike','rachel')*]
%True
%>>> [*se_conocen('rachel','harvey')*]
%False
%\end{lstlisting}
\end{itemize}
