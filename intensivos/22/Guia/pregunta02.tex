\section{Aeropuerto}

Se presentan las siguientes estructuras.
La primera es una lista de tuplas, donde cada tupla contiene 2 valores. Primero esta el código del vuelo, y el segundo valor es una tupla que contiene la fecha de salida del vuelo (formato año, mes, dia), ciudad de origen y estado actual del vuelo:

\begin{lstlisting}[style=consola]
salidas = [('LAN123', ((2013,11,2), 'NewYork', 'EMBARQUE')),
('ALGO00',((2015,12,3),'Valparaiso','EN ESPERA')),#...
('MX201', ((2013,4,28),  'Cancun', 'ARRIBADO'))]
\end{lstlisting}

Una lista de tuplas, donde cada tupla tiene 2 valores. El primer valor de la tupla es el código del vuelo, y el segundo valor de la tupla es una lista de ruts de pasajeros que se encuentran en el vuelo en cuestión.

\begin{lstlisting}[style=consola]
vuelos = [('ALGO00', ['444444-4']), 
('LAN123', ['16740623-7', '1111111-1', '555555-5']), 
# ..., 
('MX201', ['777777-7'])]
\end{lstlisting}

Finalmente, un diccionario de personas que tiene como clave el rut de un pasajero y como valor una tupla con el nombre del pasajero, ciudad de origen, fecha de nacimiento y cantidad de millas en su cuenta.

\begin{lstlisting}[style=consola]
personas = {'555555-5': ('Daniela Perez', 'Roma', (1991, 8, 17), 12000), 
'777777-7': ('Jorge Perez', 'Santiago', (1989, 2, 17), 1000), 
'444444-4': ('Edwar Lopez', 'Miami', (1900, 3, 11), 120000), 
'1111111-1': ('Sandra Lazo', 'Ibiza', (1970, 4, 14), 10000), 
'16740623-7': ('OEncina', 'NewYork', (1987, 7, 22), 62000)}
\end{lstlisting}

Se pide:
\begin{enumerate}
\item Desarrollar la función \texttt{estado\_pasajero(nombre)} que a partir del nombre del pasajero retorne una tupla con su rut, ciudad de origen y estado del vuelo. Si el pasajero no se encuentra en el sistema se deberá retornar \texttt{None}.

\begin{lstlisting}[style=consola]
>>> estado_pasajero('OEncina')
('16740623-7', 'NewYork', 'EMBARQUE')
\end{lstlisting}

\item Desarrollar la función \texttt{cambia\_de\_vuelo(rut, nuevo\_vuelo, millas)} que mueva al pasajero de su actual vuelo y lo agregue al nuevo\_vuelo. Adicionalmente se le sumaran la cantidad de millas entregada como parámetro a su cuenta. Esta operación retorna \texttt{True} si existe el pasajero y \texttt{False} en caso contrario.

\begin{lstlisting}[style=consola]
>>> cambia_de_vuelo('1111111-1','ALGO00', 5000)
True
\end{lstlisting}

\item Desarrollar la función \texttt{personas\_por\_estado(estado)} que retorna una lista de nombres de personas que se encuentran en un avión en el estado \texttt{estado}.

\begin{lstlisting}[style=consola]
>>> personas_por_estado("EMBARQUE")
['OEncina', 'Daniela Perez']
\end{lstlisting}

\item \texttt{Desarrollar la función filtro\_nac(fecha, estado)} que retorne una lista con todos los pasajeros nacidos después de \texttt{fecha} y que su vuelo este en el estado indicado.

\begin{lstlisting}[style=consola]
>>> filtro_nac((1980,1,1), 'EMBARQUE')
['555555-5, '16740623-7']
\end{lstlisting}

\end{enumerate}

Nota: debe tener en consideración de que las estructuras de datos presentadas son solo ejemplos, las reales serán mucho mas grandes, así que su programa debe ser capaz de poder trabajar con datos de tamaño indefinido que respetan la estructura presentada.