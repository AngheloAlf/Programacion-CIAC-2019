
\section{X-Espacio}

La reconocida internacionalmente empresa \texttt{X-Espacio}, la cual hace investigaciones espaciales ha encontrado señales que parecieran ser de vida inteligente que intenta comunicarse con nosotros.

Después de un arduo estudio de estas señales, se ha llegado a la conclusión de que estas señales siderales son mensajes codificados en binario, por lo que se le ha pedido a usted, experto programador, que construya un programa capaz de descifrar dichas señales.

Según esta investigación, las ondas corresponden a números binarios desde el 1 al 26, y estos corresponden a una letra de nuestro alfabeto.

Se espera que una secuencia de todos estos caracteres nos entregue frases legibles por nosotros los humanos.

Para transformar un número binario a decimal se debe seguir el siguiente procedimiento:

\begin{itemize}
    \item Se toma el ultimo numero de la secuencia. Debemos llevar un \textit{contador} y una \textit{suma}.
    \item Si este número es "1", le agregamos $2^{contador}$ a la \textit{suma}.
    \item Si el número es "0", lo ignoramos.
    \item Aumentamos en uno al contador.
    \item Repetimos los pasos recién mencionados, pero con el numero siguiente en la secuencia, es decir el que esta a la izquierda del que acabamos de usar.
    \item Nos detenemos cuando ya no queden números por procesar.
\end{itemize}

Tomemos de ejemplo el numero binario \texttt{101110}. El equivalente numérico de dicha secuencia es: $2^{1} + 2^{2} + 2^{3} + 2^{5}$, igual a 46.

Sabiendo esto, su programa debe pedir secuencias binarias hasta que se ingrese un 0, y luego mostrar el mensaje de nuestros amigos espaciales.

Para poder 

\begin{lstlisting}[style=consola]
Ingrese binario: [*101*]
Ingrese binario: [*11000*]
Ingrese binario: [*1001*]
Ingrese binario: [*10100*]
Ingrese binario: [*1111*]
Ingrese binario: [*0*]
exito
\end{lstlisting}