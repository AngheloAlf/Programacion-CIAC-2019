\section*{Resumen}

\texttt{diccionario = \{llave1:valor1, llave2:valor2, ..., llaven:valorn\}}

\begin{itemize}
    \item Colecciones que asocian un valor a una llave, son indexadas, pero el índice es la llave.
    \item Para inicializar -> \texttt{diccionario=\{\}}, \texttt{diccionario=dict()}
    \item Las llaves no se pueden repetir ni cambiar, los valores si.
    \item Para acceder a un valor:
\begin{lstlisting}[style=consola]
>>> diccionario["llave1"]
"valor1"
\end{lstlisting}

    \item Para agregar un elemento al diccionario:
\begin{lstlisting}[style=consola]
>>> diccionario[llave] = valor
\end{lstlisting}

    \item Para eliminar un par llave-valor:
\begin{lstlisting}[style=consola]
>>> del diccionario[llave]
\end{lstlisting}

    \item Si se introduce una una llave que no existe, se agrega al diccionario. Si se introduce una llave que ya existe se reemplaza el valor anterior con el ingresado.
    
    \item Existen distintas formas de recorrer los diccionarios.
    
    \begin{itemize}
        \item \texttt{for i in diccionario:} Recorre sólo las llaves.
        \item \texttt{for i in diccionario.keys():} Al igual que el anterior, recorre solo las llaves.
        \item \texttt{for i in diccionario.values():} Recorre solo los valores.
        \item \texttt{for i in diccionario.items():} Recorre una lista de tuplas de la forma \texttt{(llave, valor)}
    \end{itemize}
    
\end{itemize}
