\section{Matrices}

Este problema se presenta en la página \url{https://projecteuler.net/problem=11}. La siguiente grilla de 20x20, cuatro números se marcan diagonalmente 
\begin{table}[H]
\begin{tabular}{llllllllllllllllllll}
08 & 02 & 22 & 97 & 38 & 15 & 00 & 40                      & 00                               & 75                               & 04                               & 05                               & 07 & 78 & 52 & 12 & 50 & 77 & 91 & 08 \\
49 & 49 & 99 & 40 & 17 & 81 & 18 & 57                      & 60                               & 87                               & 17                               & 40                               & 98 & 43 & 69 & 48 & 04 & 56 & 62 & 00 \\
81 & 49 & 31 & 73 & 55 & 79 & 14 & 29                      & 93                               & 71                               & 40                               & 67                               & 53 & 88 & 30 & 03 & 49 & 13 & 36 & 65 \\
52 & 70 & 95 & 23 & 04 & 60 & 11 & 42                      & 69                               & 24                               & 68                               & 56                               & 01 & 32 & 56 & 71 & 37 & 02 & 36 & 91 \\
22 & 31 & 16 & 71 & 51 & 67 & 63 & 89                      & 41                               & 92                               & 36                               & 54                               & 22 & 40 & 40 & 28 & 66 & 33 & 13 & 80 \\
24 & 47 & 32 & 60 & 99 & 03 & 45 & 02                      & 44                               & 75                               & 33                               & 53                               & 78 & 36 & 84 & 20 & 35 & 17 & 12 & 50 \\ \cline{9-9}
32 & 98 & 81 & 28 & 64 & 23 & 67 & \multicolumn{1}{l|}{10} & \multicolumn{1}{l|}{\textbf{26}} & 38                               & 40                               & 67                               & 59 & 54 & 70 & 66 & 18 & 38 & 64 & 70 \\ \cline{9-10}
67 & 26 & 20 & 68 & 02 & 62 & 12 & 20                      & \multicolumn{1}{l|}{95}          & \multicolumn{1}{l|}{\textbf{63}} & 94                               & 39                               & 63 & 08 & 40 & 91 & 66 & 49 & 94 & 21 \\ \cline{10-11}
24 & 55 & 58 & 05 & 66 & 73 & 99 & 26                      & 97                               & \multicolumn{1}{l|}{17}          & \multicolumn{1}{l|}{\textbf{78}} & 78                               & 96 & 83 & 14 & 88 & 34 & 89 & 63 & 72 \\ \cline{11-12}
21 & 36 & 23 & 09 & 75 & 00 & 76 & 44                      & 20                               & 45                               & \multicolumn{1}{l|}{35}          & \multicolumn{1}{l|}{\textbf{14}} & 00 & 61 & 33 & 97 & 34 & 31 & 33 & 95 \\ \cline{12-12}
78 & 17 & 53 & 28 & 22 & 75 & 31 & 67                      & 15                               & 94                               & 03                               & 80                               & 04 & 62 & 16 & 14 & 09 & 53 & 56 & 92 \\
16 & 39 & 05 & 42 & 96 & 35 & 31 & 47                      & 55                               & 58                               & 88                               & 24                               & 00 & 17 & 54 & 24 & 36 & 29 & 85 & 57 \\
86 & 56 & 00 & 48 & 35 & 71 & 89 & 07                      & 05                               & 44                               & 44                               & 37                               & 44 & 60 & 21 & 58 & 51 & 54 & 17 & 58 \\
19 & 80 & 81 & 68 & 05 & 94 & 47 & 69                      & 28                               & 73                               & 92                               & 13                               & 86 & 52 & 17 & 77 & 04 & 89 & 55 & 40 \\
04 & 52 & 08 & 83 & 97 & 35 & 99 & 16                      & 07                               & 97                               & 57                               & 32                               & 16 & 26 & 26 & 79 & 33 & 27 & 98 & 66 \\
88 & 36 & 68 & 87 & 57 & 62 & 20 & 72                      & 03                               & 46                               & 33                               & 67                               & 46 & 55 & 12 & 32 & 63 & 93 & 53 & 69 \\
04 & 42 & 16 & 73 & 38 & 25 & 39 & 11                      & 24                               & 94                               & 72                               & 18                               & 08 & 46 & 29 & 32 & 40 & 62 & 76 & 36 \\
20 & 69 & 36 & 41 & 72 & 30 & 23 & 88                      & 34                               & 62                               & 99                               & 69                               & 82 & 67 & 59 & 85 & 74 & 04 & 36 & 16 \\
20 & 73 & 35 & 29 & 78 & 31 & 90 & 01                      & 74                               & 31                               & 49                               & 71                               & 48 & 86 & 81 & 16 & 23 & 57 & 05 & 54 \\
01 & 70 & 54 & 71 & 83 & 51 & 54 & 69                      & 16                               & 92                               & 33                               & 48                               & 61 & 43 & 52 & 01 & 89 & 19 & 67 & 48
\end{tabular}
\end{table}

El producto de estos números es $26 \cdot 63 \cdot 78 \cdot 14 = 1788696 $

\paragraph{Desafío:} ¿Cuál es el producto más grande de cuatro números adyacentes en la misma dirección (hacia arriba, abajo, izquierda, derecha o diagonal) en la grilla de 20x20 ?

\paragraph{Primeros pasos:} Lo recomendable en este caso es crear una matriz de la forma lista de listas, para esto nos ayudaremos de algunos códigos del certamen 3 de archivos de texto y manejo de strings, se presentan a continuación con tal de que usted pueda copiar y pegar los comandos sin necesidad aún de saber ocuparlos.

\begin{enumerate}
    \item Copie y pegue la grilla en un bloc de notas en alguna carpeta de su computador, en este caso se guardará este archivo con el nombre \texttt{grid.txt}. A la vez se creará un archivo .py en la misma carpeta anterior
    \item El código para crear la lista de listas se presenta a continuación
    
    \lstinputlisting[
    style=mypy,
    caption=\texttt{matriz.py}]{Code/matriz.py}
    \item Desarrolle la función \texttt{mayor\_fila(matriz)} que recibiendo la lista de listas \texttt{matriz} entregue el mayor producto de cuatro números adyacentes presente en todas las filas.
    \item Desarrolle la función \texttt{mayor\_columna(matriz)} que recibiendo la lista \texttt{matriz} entregue el mayor producto existente de cuatro números adyacentes en la misma columna
    \item \texttt{mayor\_diagonal(matriz)} que recibiendo la lista \texttt{matriz} retorne el producto más grande existente de cuatro números adyacentes diagonalmente.
    \item Por último desarrolle un programa que encuentre el máximo entre estos números y lo imprima por pantalla. Si todo está bien, puede ser capaz de ingresar a la página mostrada anteriormente de Project Euler y agregar su respuesta.
\end{enumerate}
