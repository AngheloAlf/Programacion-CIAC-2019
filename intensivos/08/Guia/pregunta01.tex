%%%Pregunta 1
\section{Rectas}

En el plano, una recta esta descrita por la ecuación:
\begin{center}
	$y = mx + n$
\end{center}

Teniendo esto en consideración, se puede almacenar esta ecuación en python como una lista \texttt{[m, n]}.

Sabiendo esto, cree las siguientes funciones:

\begin{enumerate}
\item \texttt{puntoEnRecta(punto, recta)}; la cual retorna \textbf{True} si el punto, en formato de lista \texttt{[x, y]}, se encuentra contenido en la recta, y \textbf{False} en caso contrario. Tener en consideración que para que un punto esté en una recta, se debe evaluar el \texttt{x} del punto en la recta y si el resultado es igual al \texttt{y} del punto, este punto se encuentra en la recta.

\begin{lstlisting}[style=consola]
>>> [*recta = [2, 5]*]
>>> [*puntoEnRecta([-1, 3], recta)*]
True
>>> [*puntoEnRecta([0, 1], recta)*]
False
\end{lstlisting}

\item \texttt{sonParalelas(recta1, recta2)}; la cual retorna \textbf{True} si las rectas son paralelas entre si, \textbf{False} en caso contrario.

\begin{lstlisting}[style=consola]
>>> [*sonParalelas([-1, 3], [5,9])*]
False
>>> [*sonParalelas([2, 8], [2, -3])*]
True
\end{lstlisting}

\item \texttt{sonPerpendiculares(recta1, recta2)}; la cual retorna \textbf{True} si las rectas son perpendiculares entre si, \textbf{False} en caso contrario. Un par de rectas son perpendiculares entre si, si la multiplicación de sus pendientes es $-1$.

\item \texttt{rectaQuePasaPor(punto1, punto2)}; la cual rectorna una tupla \texttt{(m, n)}, lo cual es la recta que pasa por ambos puntos. 
Tener en consideración que la ecuación de la recta puede representarse de la siguiente forma:
\begin{align*}
	y - y_{0} &= \frac{y_{1} - y_{0}}{x_{1} - x_{0}}(x - x_{0}) 
\end{align*} 

\begin{lstlisting}[style=consola]
>>> [*rectaQuePasaPor((2, 3), (3, 2))*]
(-1.0, 5.0)
>>> [*rectaQuePasaPor((9, 6), (-3, 9))*]
(-0.25, 8.25)
\end{lstlisting}

\item \texttt{puntoInterseccion(recta1, recta2)}; la cual retorna una tupla \texttt{(x, y)}, el cual es el punto de intersección entre las rectas, o \textbf{None} y las rectas son paralelas.

\begin{lstlisting}[style=consola]
>>> [*puntoInterseccion((-2, 12), (1, 3))*]
(3.0, 6.0)
>>> [*puntoInterseccion((3, 5), (3, -8))*]
None
\end{lstlisting}


\end{enumerate}

Nota: Recuerde que puede ocupar funciones ya realizadas al hacer una nueva función.
