\section{Ordenamiento de código}

El siguiente código esta desordenado y sin endentar, su misión sera ordenarlo de modo que el código haga lo que su descripción dice.

\begin{enumerate}

\item[$\triangle$.] Ordene la función \texttt{encriptador(archivo)}, la cual recibe el nombre a un archivo a encriptar para cambiar algunas palabras por otras e invierte las lineas. La función retorna nada.

\begin{lstlisting}[style=consola]
arch.close()
temp = open("temp.txt")
arch.close()
linea = linea.replace("toma", "junta")
temp = open("temp.txt", "w")
temp.write(linea+"\n")
linea = linea.strip()
return
linea = linea[::-1]
def encriptador(archivo):
for linea in arch:
arch = open(archivo, "w")
arch = open(archivo)
linea = linea.replace("paro", "jugar")
temp.close()
linea = linea.replace("edificio A", "mall")
temp.close()
arch.write(linea)
for linea in temp:
\end{lstlisting}


\item[$\forall$.] La función \texttt{obtenerMayor(archivo)}, recibe el nombre de un archivo, el cual cada linea esta separada por \texttt{;}. Lee el archivo linea a linea, obtiene un nombre que esta en la primera parte de la linea, y abre un archivo con ese nombre. Este nuevo archivo esta separado por \texttt{-}, donde cada linea contiene un par \texttt{numero}, \texttt{palabra}. Finalmente busca el \texttt{numero} mas grande y entrega la \texttt{palabra} asociada a este numero.

\begin{lstlisting}[style=consola]
mayorNum = -float("inf")
mayorPalabra = palabra
archNombre.close()
mayorPalabra = ""
num = float(num)
arch.close()
archNombre = open(nombre+".txt")
num = mayorNum
for linea2 in archNombre:
nombre = linea[0]
arch = open(archivo)
num, palabra = linea2.strip().split("-")
if num > mayorNum:
def obtenerMayor(archivo):
for linea in arch:
return mayorPalabra
linea = linea.strip().split(";")
\end{lstlisting}

\end{enumerate}
