\section*{Conociendo el mundo de los \texttt{Diccionarios}}

\begin{itemize}
\item Un diccionario es un tipo de dato desordenado el cual permite asociar pares de valores, uno siendo la ``llave'' y el otro su ``valor''

\item Para crear un diccionario vació puede usar \{ \} o la función \textit{dict()}. Para crear un diccionario con elementos por defecto, debe hacerlo con \{ \} y separando cada llave y su valor asociado con \textit{:}
\begin{lstlisting}[style=consola]
>>> a = {}
>>> b = dict()
>>> c = {"13345678-3": "Miguel", "14062073-4": "Anghelo"}
\end{lstlisting}


\item Para obtener un valor que esta en la llave \textit{l} de un diccionario \textit{d}, debe usar $d[l]$ :
\begin{lstlisting}[style=consola]
>>> print c["13345678-3"]
'Miguel'
\end{lstlisting}

\item Para agregar elementos, simplemente se asigna el valor a la llave correspondiente:
\begin{lstlisting}[style=consola]
>>> c["17064073-k"] = "Francisco"
>>> c["14062073-4"] = "Persona"
\end{lstlisting}

Cabe destacar que si se asigna un valor a una llave que ya tenia un elemento, el elemento viejo se elimina y se guarda el nuevo (No se pueden tener llaves repetidas).

\item Para borrar una llave:
\begin{lstlisting}[style=consola]
>>> del c["17064073-k"]
\end{lstlisting}

\item Se puede usar el ciclo \textit{for}, donde la variable \textit{i} sera las llaves del diccionario:
\begin{lstlisting}[style=consola]
>>> for i in c:
        print i
14062073-4
13345678-3
\end{lstlisting}

\item Para iterar según los valores y no las llaves, se usa \textit{diccionario.values()}
\begin{lstlisting}[style=consola]
>>> for i in c.values():
        print i
Miguel
Persona
\end{lstlisting}

\item Para iterar según las llaves y los valores a la vez, se usa \textit{diccionario.items()}:
\begin{lstlisting}[style=consola]
>>> for l, v in c.items():
        print "El rut de", v, "es: ", l
El rut de Persona es: 14062073-4
El rut de Miguel es: 13345678-3
\end{lstlisting}        

\item El operador \textit{n in d} indica si la variable \textit{n} es una llave del diccionario \textit{d}
\begin{lstlisting}[style=consola]
>>> "14062073-4" in c
True
\end{lstlisting}
\end{itemize}
