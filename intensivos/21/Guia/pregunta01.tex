\section{Regiones}

Desde 2018, geográficamente Chile está dividido en 16 regiones desde Arica hasta la Antártica Chilena. Además de la división geográfica, también existen las regiones naturales (zonas), las cuales son cinco: Norte grande, Norte chico, Zona central, Zona sur, Zona austral.

La tarea que se le encomienda a usted es construir ciertas funciones que ayuden a obtener información con respecto a estas divisiones. 

Para esto, usted cuenta con la información de cada región en la lista de tuplas \texttt{regiones}, donde cada tupla contiene 5 elementos, el número de región (en romano), el nombre, la cantidad de habitantes, la superficie en $km^2$ y la capital. Guiese por este ejemplo:

\begin{lstlisting}[style=consola]
regiones = [
# (numero, nombre, poblacion, superficie, capital)
  ("XV",   "Region de Arica y Parinacota", 226068,  16873.3, "Arica"), 
  ("I",    "Region de Tarapaca",           581802,  42225.8, "Iquique"), 
  ("II",   "Region de Antofagasta",        607534, 126049.1, "Antofagasta"),
  ("III",  "Region de Atacama",            286168,  75176.2, "Copiapo"),
  ("IV",   "Region de Coquimbo",            757586,  40579.9, "La Serena"),
  ("V",    "Region de Valparaiso",        1815902,  16396.1, "Valparaiso"),
  ("RM",   "Region Metropolitana",        7112808,  15403.2, "Santiago"),
  ("VI",   "Region de O'Higgins",          914555,  16387.0, "Rancagua"),
  ("VII",  "Region del Maule",            1044950,  30296.1, "Talca"),
  ("XVI",  "Region de Nuble",              480609,  13178.5, "Chillan"),
  ("VIII", "Region del Biobio",           1556805,  23890.2, "Concepcion"),
  ("IX",   "Region de La Araucania",       957224,  31842.3, "Temuco"),
  ("XIV",  "Region de Los Rios",           384837,  18429.5, "Valdivia"),
  ("X",    "Region de Los Lagos",          828708,  48583.6, "Puerto Montt"),
  ("XI",   "Region de Aysen",              103158, 108494.4, "Coyhaique"),
  ("XII",  "Region de Magallanes",         166533, 132297.2, "Punta Arenas")
]
\end{lstlisting}

Además, usted cuenta con la lista de tuplas \texttt{zonas}, donde cada tupla esta compuesta por 2 elementos, el nombre de la zona y una lista que contiene los números de las regiones que corresponden a dicha zona.

\begin{lstlisting}[style=consola]
zonas = [
  ("norte grande", ["XV", "I", "II"]),
  ("norte chico",  ["III", "IV"]), 
  ("zona central", ["V", "RM", "VI", "VII", "XVI", "VIII"]),
  ("zona sur",     ["IX", "XIV", "X"]),
  ("zona austral", ["XI", "XII"])
]
\end{lstlisting}

Escriba las siguientes funciones:
\begin{enumerate}
    \item \texttt{buscar\_region(regiones, numero)}; la cual recibe la lista de regiones y un string que contiene el numero (en romano) de la región que queremos buscar. Entrega la posición en la que se encuentra dicha región con respecto a la lista \texttt{regiones} y la tupla asociada a dicha región en particular.
    
\begin{lstlisting}[style=consola]
>>> [*print(buscar_region(regiones, "XI"))*]
(14, ('XI', 'Region de Aysen', 103158, 108494.4, 'Coyhaique'))
\end{lstlisting}

    \item \texttt{densidad\_por\_zona(regiones, zonas)}; la cual entrega una lista de tuplas, donde el primer elemento de cada tupla es el nombre de la zona y el segundo valor es la densidad poblacional de dicha zona. La densidad poblacional se calcula considerando el total de la población dividido en la totalidad de la superficie.
    
\begin{lstlisting}[style=consola]
>>> [*lista_densidades = densidad_por_zona(regiones, zonas)*]
>>> [*print(lista_densidades)*]
[('norte grande', 7.644708401161879), ('norte chico', 9.016837989531437), 
('zona central', 111.86071789883438), ('zona sur', 21.95903309277996), 
('zona austral', 1.120018306286432)]
\end{lstlisting}
    
    \item Como bien sabemos, las regiones no se mantienen constantes en el tiempo, puede cambiar su población o, incluso, su superficie, entre otras cosas. 
    
    Dado esto, se le pide escribir la función \texttt{cambiar\_region(regiones, numero, datos)}; la cual recibe la lista de regiones, el string \texttt{numero} el cual indica el numero de la región (en romano) y una tupla de 4 datos que contiene el nombre, población, superficie y capital de la región. Esta función modifica a la región en cuestión con los datos nuevos, manteniendo el orden original de la lista.
    
\begin{lstlisting}[style=consola]
>>> [*cambiar_region(regiones, "I", ("Region de ", 750000, 42225.8, "Iquique"))*]
>>> [*print(regiones[0:3])*]
[('XV', 'Region de Arica y Parinacota', 226068, 16873.3, 'Arica'), 
('I', 'Region de Tarapaca', 750000, 42225.8, 'Iquique'), 
('II', 'Region de Antofagasta', 607534, 126049.1, 'Antofagasta')]
\end{lstlisting}

    \item Como también sabemos, con el tiempo se van creando nuevas regiones en base a regiones ya existentes, disminuyendo la población y superficie de la región desde la que se esta creando. Sabiendo esto, se le pide la función \texttt{crear\_region(regiones, zonas, nueva\_region, numero\_antigua)} la cual recibe las 2 listas antes descritas, la tupla \texttt{nueva\_region}, la cual es la región nueva que se va a agregar y tiene la misma estructura que una tupla de la lista \texttt{regiones}; y finalmente el string \texttt{numero\_antigua}, el cual indica la región que se esta usando para crear esta nueva región.
    
    Esta función debe:
    \begin{itemize}
        \item Poner la región nueva justo debajo de la región antigua en la lista \texttt{regiones}.
        \item Disminuir la población y superficie que tendrá la región nueva desde la región antigua. Por ejemplo, si creamos una nueva región, con 200000 habitantes y superficie 10000 $km^2$, en base a la \textit{Region de los Lagos}, debemos buscar dicha región, disminuir su población y superficie en esas cantidades, y actualizar la información en la lista \texttt{regiones}.
        \item Agregar la región nuevo a la misma zona donde se encuentra la región antigua.
    \end{itemize} 
    
\begin{lstlisting}[style=consola]
>>> [*nueva_region = ("XVII", "Republica independiente de Placeres", 2790, 30,
"USM")*]
>>> [*crear_region(regiones, zonas, nueva_region, "V")*]
>>> [*print(regiones[5:8])*]
[('V', 'Region de Valparaiso', 1813112, 16366.1, 'Valparaiso'), 
('XVII', 'Republica independiente de Placeres', 2790, 30, 'USM'), 
('RM', 'Region Metropolitana', 7112808, 15403.2, 'Santiago')]
\end{lstlisting}

\end{enumerate}
