\section{Notas, archivos y ¿diccionarios?}

Considerando el mismo formato de la pregunta anterior, cree las siguientes funciones:

\begin{itemize}
    %%% Se necesitaron 4 cabezas para pensar en esta palabra.
    \item \texttt{diccioinador(archivo)}. Esta función recibe el nombre de un archivo con el formato entregado en la pregunta anterior. 
    
    Retorna un diccionario, donde las llaves son el nombre de las personas, y el valor es un diccionario de ramos de dicho estudiante. La llave de esta diccionario es la sigla de un ramo, y la llave es una tupla con las notas de este ramo como enteros (no se considera el promedio de notas).
    
    \begin{lstlisting}[style=consola]
>>> [*dicc = diccioinador("notas.txt")*]
>>> [*print(dicc)*]
{'anghelo': {'mat021': (76, 78, 54)}, 
 'miguel': {'mat021': (80, 56, 67), 'fis100': (95, 86, 75)}, 
 'diego': {'iwi131': (100, 99, 98)}
 }
    \end{lstlisting}
    
    \item \texttt{archivinador(archivo, diccionario)}. Esta función recibe el nombre de un archivo y el diccionario antes descrito. 
    
    Esta función escribe en el archivo los datos del diccionario, respetando la estructura del archivo antes descrito.
    
    \begin{lstlisting}[style=consola]
>>> [*dicc = archivinador("notas2.txt", dicc)*]
>>> 
    \end{lstlisting}
    
\begin{center}
\begin{tabular}{|l|}
    \hline
    \texttt{notas2.txt} \\ 
    \hline
    miguel\#mat021\#80;56;67 \\
    miguel\#fis100\#95;86;75 \\
    anghelo\#mat021\#76;78;54 \\
    diego\#iwi131\#100;99;98 \\
    \hline
\end{tabular}
\end{center}
    
    \item \texttt{agregar\_alumno\_inador(archivo, nombre, ramos)}. Esta función recibe el nombre de un archivo con la estructura antes dada, un string con el nombre de un alumno y un diccionario, donde las llaves del diccionario son las siglas de los ramos, y las llaves son una tupla con las notas de dicho estudiante.
    
    Esta función agrega los ramos en el diccionario \texttt{ramos} de estudiante \texttt{nombre} al archivo en cuestión. Si el estudiante ya tenia ese ramo, entonces se debe reemplazar las notas de ese ramo en el archivo.
    
    Lógicamente, si el estudiante no existía en el archivo anteriormente, debe ser agregado.
    
    \begin{lstlisting}[style=consola]
>>> [*ramos = {"iwi131": (42, 69, 90), "fis100": (51, 63, 54)}*]
>>> [*agregar_alumno_inador("notas2.txt", "anghelo", ramos)*]
>>> 
    \end{lstlisting}

\begin{center}
\begin{tabular}{|l|}
    \hline
    \texttt{notas2.txt} \\ 
    \hline
    anghelo\#mat021\#76;78;54 \\
    anghelo\#iwi131\#42;69;90 \\
    anghelo\#fis100\#51;63;54 \\
    miguel\#mat021\#80;56;67 \\
    miguel\#fis100\#95;86;75 \\
    diego\#iwi131\#100;99;98 \\
    \hline
\end{tabular}
\end{center}
\end{itemize}
