\section*{Resumenario de archivos}

\begin{itemize}
    \item Un archivo de texto es un archivo que esta compuesto únicamente de texto plano, sin formato, solo caracteres (letras, números, y símbolos varios como \%).

    \item Para abrir un archivo, se utiliza la función \texttt{open(archivo, modo)}, donde el parámetro \texttt{archivo} es el nombre del archivo a abrir, y \texttt{modo} es un string indicando en que modo se va a abrir el archivo (lectura, escritura, etc.). Este ultimo parámetro puede ser omitido, en tal caso el archivo se esta abriendo en modo de lectura.
    
    \item Se pueden abrir archivos de múltiples maneras, pero en este curso tan solo se utilizaran 3. "\texttt{r}" (\textit{read}) para leer el archivo, "\texttt{w}" (\textit{write}) para escribir en el archivo, y "\texttt{a}" (\textit{append}) para escribir al final.
    
    \item close
    
    \item write
    
    \item (ojo, modo "\texttt{w}" borra los contenidos del archivo anterior)
    
    \item for
    
    \item explicar append
    
    \item .split().strip() y separadores varios
    
    \item archivos temporales (?)
    
\end{itemize}
