\section*{Cálculo de $\pi$}

Desarrolle un programa que le permita calcular una aproximación de $\pi$ usando la siguiente suma infinita

\begin{equation}
    \pi=4\cdot \left( 1-\frac{1}{3}+\frac{1}{5}-\frac{1}{7} + \ldots \right)
\end{equation}

El funcionamiento debe ser como lo muestran los siguientes ejemplos

\begin{lstlisting}[style=consola]
n: 3
3.466666666666667
\end{lstlisting}

\begin{lstlisting}[style=consola]
n: 3
3.466666666666667
\end{lstlisting}
\pagebreak[4]
\section*{Digitos ordenados}

Escriba un programa que reciba números e imprima si cada dígito de estos números está ordenado de menor a mayor. El programa se acaba si el usuario escribe la palabra \texttt{Fin}

\begin{lstlisting}[style=consola]
Ingrese un numero: [*15328*]
El numero 15328 no esta ordenado
Ingrese un numero: [*1589*]
El numero 1589 esta ordenado
Ingrese un numero: [*35669*]
El numero 35669 esta ordenado
Ingrese un numero: [*16423*]
El numero 16423 no esta ordenado
Ingrese un numero: [*111123566689*]
El numero 111123566689 esta ordenado
Ingrese un numero: [*Fin*]
\end{lstlisting}

