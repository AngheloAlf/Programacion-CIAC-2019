\section{Cálculo matricial programable}

Una matriz es una forma de agrupar datos bidimensionalmente. Estas se escriben de la siguiente forma:
\begin{equation*}
    M=\left[
        \begin{array}{ccc}
            1 & 2 & 6 \\
            3 & 4 & 22 \\
        \end{array}
    \right]
\end{equation*}

Una matriz puede entenderse como muchos vectores juntos, o como una lista con muchas listas. La matriz anterior escrita como lista de listas seria:
\begin{lstlisting}[style=consola]
M = [[1, 2, 6], [3, 4 ,22]]
\end{lstlisting}
Donde cada lista interior se corresponde con una fila de la matriz.

La matriz también tiene algunas operaciones, como la suma y resta, la multiplicación por escalares y la traspuesta.

La adición (y sustracción) de matrices esta dado por:
\begin{equation*}
    \left[
        \begin{array}{ccc}
            a_0 & a_1 & a_2 \\
            a_3 & a_4 & a_5 \\
            a_6 & a_7 & a_8 \\
        \end{array}
    \right]
    \pm
    \left[
        \begin{array}{ccc}
            a_9 & a_{10} & a_{11} \\
            a_{12} & a_{13} & a_{14} \\
            a_{15} & a_{16} & a_{17} \\
        \end{array}
    \right]
    =
    \left[
        \begin{array}{ccc}
            a_0 \pm a_9 & a_1 \pm a_{10} & a_2 \pm a_{11} \\
            a_3 \pm a_{12} & a_4 \pm a_{13} & a_5 \pm a_{14} \\
            a_6 \pm a_{15} & a_7 \pm a_{16} & a_8 \pm a_{17} \\
        \end{array}
    \right]
\end{equation*}

La multiplicación de matrices por un escalar (número) esta dado por:
\begin{equation*}
    x*
    \left[
        \begin{array}{ccc}
            a_0 & a_1 & a_2 \\
            a_3 & a_4 & a_5 \\
            a_6 & a_7 & a_8 \\
        \end{array}
    \right]
    =
    \left[
        \begin{array}{ccc}
            x*a_0 & x*a_1 & x*a_2 \\
            x*a_3 & x*a_4 & x*a_5 \\
            x*a_6 & x*a_7 & x*a_8 \\
        \end{array}
    \right]
\end{equation*}

Finalmente la traspuesta de una matriz esta dado como:
\begin{equation*}
    \left[
        \begin{array}{ccc}
            a_0 & a_1 & a_2 \\
            a_3 & a_4 & a_5 \\
        \end{array}
    \right]
    =
    \left[
        \begin{array}{cc}
            a_0 & a_3 \\
            a_1 & a_4 \\
            a_2 & a_5 \\
        \end{array}
    \right]
\end{equation*}

Sabiendo todo lo anterior, se le pide:
\begin{enumerate}
    \item Una función \texttt{producto\_escalar(matriz, escalar}. Retorna la multiplicación de la matriz por el número.
    \begin{lstlisting}[style=consola]
>>> [*m = [[5, 7, 8], [4, 3, 2]]*]
>>> [*producto_punto(m, 0.5)*]
[[2.5, 3.5, 4.0], [2.0, 1.5, 1.0]]
    \end{lstlisting}
    
    \item[b.] Una función \texttt{suma(matriz\_1, matriz\_2)} que retorna la suma de las 2 matrices. Si las matrices son de distinta dimension, retorna \texttt{None}.
    \begin{lstlisting}[style=consola]
>>> [*m1 = [[1, 2, 6], [3, 4, 22]]*]
>>> [*m2 = [[5, 7, 8], [4, 3, 2]]*]
>>> [*suma(m1, m2)*]
[[6, 9, 14], [7, 7, 24]]
    \end{lstlisting}
    
    \item[$\gamma$.] Una función \texttt{resta(matriz1, matriz2)}. Retorna el resultado de la resta entre la matriz1 y la matriz2. Tenga en consideración que puede usar funciones anteriores.
    \begin{lstlisting}[style=consola]
>>> [*m1 = [[1, 2, 6], [3, 4, 22]]*]
>>> [*m2 = [[5, 7, 8], [4, 3, 2]]*]
>>> [*resta(m1, m2)*]
[[-4, -5, -2], [-1, 1, 20]]
    \end{lstlisting}
    
    \item[$\square$.] Una función \texttt{traspuesta(matriz)}. Retorna la traspuesta de la matriz.
    \begin{lstlisting}[style=consola]
>>> [*traspuesta([[1, 2, 6], [3, 4, 22]])*]
[[1, 3], [2, 4], [6, 22]]
    \end{lstlisting}
    
\end{enumerate}
