\section*{Resumen, Recetario, Formulario, Paltario, etc}

\subsection*{Tuplas}

\begin{itemize}
    \item Para inicializar una tupla use:
\begin{lstlisting}[style=consola]
tupla=(datos) #o bien
tupla=tuple(datos)
\end{lstlisting}
    \item Las tuplas no se pueden modificar una vez creadas, sólo reemplazándolas en la variable original. No tienen funciones propias (son inmutables).
    \item Se pueden desempaquetar, o asignar variables a sus elementos, por ejemplo
\begin{lstlisting}[style=consola]
>>>(nombre,apellido)=('Miguel','Godoy')
>>>nombre
'Miguel'
>>>apellido
'Godoy'
\end{lstlisting}
    \item Se pueden comparar
\begin{lstlisting}[style=consola]
>>>(1,2,3)>(1,2)
True
>>>(2,4,5)<(2,5,6)
True
>>>(1,2)==(2-1,10\%8)
True    
\end{lstlisting}
    \item Se pueden concatenar
\begin{lstlisting}[style=consola]
>>>(1,2,3)+(0,0,1)
(1,2,3,0,0,1)
\end{lstlisting}
\end{itemize}

\subsection*{Listas}

\begin{itemize}
    \item Se pueden comparar y concatenar de la misma forma que las tuplas
    \item Para inicializarlas
\begin{lstlisting}[style=consola]
lista=[] #o bien
lista=list()
\end{lstlisting}
    \item Para agregar elementos
\begin{lstlisting}[style=consola]
lista.append(dato) #o si lista ya esta creada
lista=lista+[dato]
\end{lstlisting}
 
    \item Para eliminar un dato de la lista
\begin{lstlisting}[style=consola]
lista.remove(elemento) #Conociendo el elemento a eliminar
del lista[indice] #Conociendo el indice del elemento a eliminar
\end{lstlisting}

    \item Algunos comandos varios son 
\begin{lstlisting}[style=consola]
lista.sort() #Ordena de menor a mayor la lista, RETORNA NONE
lista.reverse() #Invierte el orden de los elementos RETORNA NONE
\end{lstlisting}

    \item Tener cuidado con la asignación de una variable para una misma lista

\begin{lstlisting}[style=consola]
>>> lis_a=[1,3,2]
>>> lis_b=lis_a
>>> lis_b.remove(3)
>>> lis_a
[1, 2]
>>> lis_b
[1, 2]
\end{lstlisting}
Lo anterior pasa porque se asignan dos variables al mismo espacio de memoria, para evitar esto se hace
\begin{lstlisting}[style=consola]
>>> lis_a=[1,3,2]
>>> lis_b=list(lis_a) #Se asegura que sea una lista nueva
>>> lis_b.remove(3)
>>> lis_a
[1, 3, 2]
>>> lis_b
[1, 2]
\end{lstlisting}

\end{itemize}

\subsection*{Ciclo for}

Anteriormente para recorrer una estructura de tipo string, nos aprovechábamos de que podemos acceder a cada caracter usando un índice de la forma:

\begin{lstlisting}[style=consola]
cadena='Programacion'
contador=0
while contador<len(cadena):
    print cadena[contador],
    contador+=1
\end{lstlisting}

Que imprime
\begin{lstlisting}[style=consola]
P r o g r a m a c i o n
\end{lstlisting}

Lo anterior se puede seguir usando, pero existe una forma más compacta de recorrer estructuras usando el comando \texttt{for}

\begin{lstlisting}[style=consola]
cadena='Programacion'
for caracter in cadena:
    print caracter,
\end{lstlisting}

Que imprime lo mismo.

Para recorrer una estructura de listas con tuplas que deben desempaquetarse, como por ejemplo
\begin{lstlisting}[style=consola]
capitales=[('Chile','Santiago'),('Argentina','Bs. Aires'),('Alemania','Berlin')]
\end{lstlisting}
Se puede
\begin{itemize}
    \item[1.] Desempaquetar inmediatamente 
    \begin{lstlisting}[style=consola]
for pais,capital in capitales:
    print 'La capital de',pais,'es',capital
    \end{lstlisting}
    
    \item[2.] Desempaquetar en un paso siguiente
    \begin{lstlisting}[style=consola]
for tupla in capitales:
    pais,capital=tupla
    print 'La capital de',pais,'es',capital
    \end{lstlisting}
    
    \item[3.] No desempaquetar, y usar índices para referirse a los datos
    \begin{lstlisting}[style=consola]
for tupla in capitales:
    print 'La capital de',tupla[0],'es',tupla[1]
    \end{lstlisting}
\end{itemize}