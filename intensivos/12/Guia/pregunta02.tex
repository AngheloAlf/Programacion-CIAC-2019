\section{Signos zodiacales}
Se tiene una lista \texttt{fechas}, la cual contiene tuplas, cada tupla tiene el nombre de una persona y una tupla de la forma \texttt{anio,mes,dia} que indica la fecha de nacimiento del personaje.

\begin{lstlisting}[style=consola]
fechas=[
    ('Mike',(1994,7,19)),
    ('Gary',(1989,5,22)),
    ('Brad',(1975,5,22)),
    ('Angie',(1984,5,22)),
    ('Peter',(1967,12,4)),
    ('Larry',(2001,3,14)),
    ('Moe',(2000,12,4))]
\end{lstlisting}

También se tiene una estructura de signos zodiacales

\begin{lstlisting}[style=consola]
signos = [
   ('aries',       (( 3, 21), ( 4, 20))),
   ('tauro',       (( 4, 21), ( 5, 21))),
   ('geminis',     (( 5, 22), ( 6, 21))), 
   ('cancer',      (( 6, 22), ( 7, 23))),
   ('leo',         (( 7, 24), ( 8, 23))),  
   ('virgo',       (( 8, 24), ( 9, 23))),
   ('libra',       (( 9, 24), (10, 23))), 
   ('escorpio',    ((10, 24), (11, 22))),
   ('sagitario',   ((11, 23), (12, 21))),  
   ('capricornio', ((12, 22), ( 1, 20))),
   ('acuario',     (( 1, 21), ( 2, 19))),  
   ('piscis',      (( 2, 20), ( 3, 20)))]
\end{lstlisting}

Y una lista con los elementos de los distintos signos del zodiaco.
\begin{lstlisting}[style=consola]
elementos=[
    ('tierra',('tauro','virgo','capricornio')),
    ('fuego',('aries','leo','sagitario')),
    ('aire',('geminis','libra','acuario')),
    ('agua',('cancer','escorpion','piscis'))]
\end{lstlisting}

Se le pide a usted programar las funciones
\begin{itemize}
    \item \texttt{determinar\_signo(signos, fecha)} que recibiendo la lista de signos y una tupla con la fecha de forma \texttt{anio,mes,dia} retorne un string con el signo zodiacal correspondiente.
\begin{lstlisting}[style=consola]
>>> [*signo = determinar_signo(signos, (1994,7,19))*]
>>> [*print(signo)*]
cancer
>>> [*signo = determinar_signo(signos, (1996,12,27))*]
>>> [*print(signo)*]
capricornio

\end{lstlisting}
    \item \texttt{nombre\_signo(signos, fechas)} que reciba la lista de signos y la lista de fechas de \texttt{fechas} y retorne una lista de tuplas que asocie el nombre de una persona con su signo zodiacal respectivo en tuplas.
\begin{lstlisting}[style=consola]
>>> [*print(nombre_signo(signos, fechas))*]
[('Mike', 'cancer'), ('Gary', 'geminis'), ('Brad', 'geminis'), 
('Angie', 'geminis'), ('Peter', 'sagitario'), ('Larry', 'piscis'), 
('Moe', 'sagitario')]
\end{lstlisting}

    \item \texttt{buscar\_elemento(lista\_elementos, signo)} que reciba un string de un signo zodiacal y retorne el elemento correspondiente (tierra, fuego, aire o agua).
\begin{lstlisting}[style=consola]
>>> [*print(buscar_elemento(elementos, 'sagitario'))*]
fuego
\end{lstlisting}

    \item \texttt{compatibilidad(lista\_elementos, fechas)} que reciba la lista \texttt{fechas} y retorne una lista que tenga tuplas, donde cada una contiene un elemento zodiacal y listas que asocien todos los nombres que pertenezcan al elemento correspondiente.
\begin{lstlisting}[style=consola]
>>> [*print(compatibilidad(elementos, fechas))*]
[('agua', ['Mike', 'Larry']), ('aire', ['Gary', 'Brad', 'Angie']), 
('fuego', ['Peter', 'Moe'])]
\end{lstlisting}

\end{itemize}
