\section{Copa americana}

Por fin ha llegado el evento deportivo mas esperado por los chilenos, la \textit{Copa americana}.

Como usted no es un gran hincha de la selección, pero quiere celebrar con sus amigos, ha decidido crear un programa que lo ayude a comprender que esta pasando con la \textit{copa}.

En base a sus investigaciones, la copa funciona en base a 3 grupos de 4 equipos (donde invitan a 2 países extranjeros).

Cada país tiene los siguientes datos: cantidad de partidos jugados, ganados, empatados, perdidos, goles a favor, goles en contra, diferencia de goles, puntos. Donde cantidad de partidos jugados, ganados, empatados, perdidos es lo que su nombre indica. La cantidad de goles a favor y goles en contra indica la cantidad total de goles que este país ha echo y le han echo durante la copa. La diferencia de goles es la resta de estos últimos 2 valores. Finalmente los puntos es la cantidad de puntos que los países obtienen por cada partido. Ganar un partido da 3 puntos, empatarlo da 1, y perderlo da 0 puntos.

Para ayudarse en esto, se ha planteado las siguientes estructuras:

\begin{lstlisting}[style=consola]
grupos = [
    # (letraGrupo, listaEquipos)
    ("A", ["Brasil", "Peru", "Venezuela", "Bolivia"]),
    ("B", ["Colombia", "Paraguay", "Catar", "Argentina"]),
    ("C", ["Chile", "Uruguay", "Ecuador", "Japon"]) ]
\end{lstlisting}

\begin{lstlisting}[style=consola]
detalles_puntaje = [
    # (Pais, partidosjugados, ganados, empatados, perdidos, goles a favor, 
    #  goles en contra, diferenciagoles, puntos)
    ("Argentina", 0, 0, 0, 0, 0, 0, 0, 0),
    ("Brasil", 0, 0, 0, 0, 0, 0, 0, 0),
    ("Bolivia", 0, 0, 0, 0, 0, 0, 0, 0),
    ("Catar", 0, 0, 0, 0, 0, 0, 0, 0),
    ("Chile", 0, 0, 0, 0, 0, 0, 0, 0),
    ("Colombia", 0, 0, 0, 0, 0, 0, 0, 0),
    ("Ecuador", 0, 0, 0, 0, 0, 0, 0, 0),
    ("Japon", 0, 0, 0, 0, 0, 0, 0, 0),
    ("Paraguay", 0, 0, 0, 0, 0, 0, 0, 0),
    ("Peru", 0, 0, 0, 0, 0, 0, 0, 0),
    ("Uruguay", 0, 0, 0, 0, 0, 0, 0, 0),
    ("Venezuela", 0, 0, 0, 0, 0, 0, 0, 0) ]
\end{lstlisting}

Finalmente, se ha planteado realizar las siguientes funciones:

\begin{enumerate}
    \item[$\star$] \texttt{mismo\_grupo(grupos, pais1, pais2)}, la cual recibe la lista de grupos antes descrita, y 2 strings con nombres de 2 países. Esta función retorna \texttt{True} si \texttt{pais1} y \texttt{pais2} están en el mismo grupo, y \texttt{False} en caso contrario.

\begin{lstlisting}[style=consola]
>>> [*print(mismo_grupo(grupos, "Chile", "Japon"))*]
True
>>> [*print(mismo_grupo(grupos, "Brasil", "Argentina"))*]
False
\end{lstlisting}

    
    \item[$\Omega$] \texttt{jugar\_partido(grupos, puntaje, pais1, pais2, goles1, goles2)}. Esta función recibe ambas listas antes descritas, el nombre de 2 países (\texttt{pais1} y  \texttt{pais2}) y un par de enteros, cada uno indicando cuantos goles hizo cada país (\texttt{goles1} corresponde a los goles que hizo \texttt{pais1}).
    
    Esta función debe actualizar los datos de ambos países de forma apropiada. Ojo que si los 2 países ingresados no están en el mismo equipo, es imposible jugar el partido, por lo que la función no debe modificar ningún dato.
    
    La función retorna \texttt{None} (osea, no retorna ningun valor).
    

\begin{lstlisting}[style=consola]
>>> [*jugar_partido(grupos, detalles_puntaje, "Brasil", "Bolivia", 3, 0)*]
>>> [*jugar_partido(grupos, detalles_puntaje, "Venezuela", "Peru", 0, 0)*]
>>> [*jugar_partido(grupos, detalles_puntaje, "Argentina", "Colombia", 0, 2)*]
>>> [*jugar_partido(grupos, detalles_puntaje, "Paraguay", "Catar", 2, 2)*]
>>> [*jugar_partido(grupos, detalles_puntaje, "Uruguay", "Ecuador", 4, 0)*]
>>> [*jugar_partido(grupos, detalles_puntaje, "Japon", "Chile", 0, 4)*]
>>> [*print(detalles_puntaje)*]
[('Brasil', 1, 1, 0, 0, 3, 0, 3, 3), ('Bolivia', 1, 0, 0, 1, 0, 3, -3, 0),
('Venezuela', 1, 0, 1, 0, 0, 0, 0, 1), ('Peru', 1, 0, 1, 0, 0, 0, 0, 1),
('Argentina', 1, 0, 0, 1, 0, 2, -2, 0), ('Colombia', 1, 1, 0, 0, 2, 0, 2, 3), 
('Paraguay', 1, 0, 1, 0, 2, 2, 0, 1), ('Catar', 1, 0, 1, 0, 2, 2, 0, 1),
('Uruguay', 1, 1, 0, 0, 4, 0, 4, 3), ('Ecuador', 1, 0, 0, 1, 0, 4, -4, 0),
('Japon', 1, 0, 0, 1, 0, 4, -4, 0), ('Chile', 1, 1, 0, 0, 4, 0, 4, 3)]
\end{lstlisting}
    
    \item[$\alpha$] \texttt{datos\_grupo(grupos, puntaje, letra\_grupo)}. Esta funcion recibe las listas anteriormente descritas y un string que contiene la letra del grupo.
    
    Esta función retorna una tupla con los datos de los países correspondientes al grupo \texttt{letra\_grupo}, pero ordenados de mayor a menor según la cantidad de puntos que tenga cada país. Si la cantidad de puntos es igual, se debe ordenar por la diferencia de goles, de mayor a menor. Si esta también es igual, puede usar el criterio que encuentre conveniente para ordenarlos.
    
\begin{lstlisting}[style=consola]
>>> [*print(datos_grupo(grupos, detalles_puntaje, "C"))*]
(('Uruguay', 1, 1, 0, 0, 4, 0, 4, 3), ('Chile', 1, 1, 0, 0, 4, 0, 4, 3),
('Japon', 1, 0, 0, 1, 0, 4, -4, 0), ('Ecuador', 1, 0, 0, 1, 0, 4, -4, 0))
\end{lstlisting}


    \item[$\&$] \texttt{cuartos\_de\_final(grupos, puntaje)}, la cual recibe ambas listas antes descritas. Esta función retorna una lista de 4 tuplas, donde cada tupla contiene 2 strings, los cuales son los nombres de los paises a enfrentarse en cuartos de final.
    
    El orden de los partidos es:
    \begin{itemize}
        \item Primer partido: Primer lugar del grupo A contra el mejor tercer lugar de los otros grupos.
        \item Segundo partido: Segundo lugar del grupo A contra segundo lugar del grupo C.
        \item Tercer partido: Primer lugar del grupo B contra el mejor segundo-tercer lugar de los otros grupos.
        \item Cuarto partido: Primer lugar del grupo C contra segundo lugar del grupo A.
    \end{itemize}
    
    Puede usar su método preferido para definir a los mejores terceros lugares.

\begin{lstlisting}[style=consola]
>>> [*print(cuartos_de_final(grupos, detalles_puntaje))*]
[('Brasil', 'Peru'), ('Venezuela', 'Chile'), ('Colombia', 'Catar'),
('Uruguay', 'Paraguay')]
\end{lstlisting}

\end{enumerate}

$._\text{\sout{¡\textit{Somos el mejor país de Chile}!}}$
