\section{Letras en orden alfabético}
Cada letra como string tiene un número asociado que se usa para comparar alfabéticamente palabras o letras, por ejemplo, al escribir en Python \texttt{ord('a')} la consola imprime \texttt{97}, lo mismo con \texttt{ord('b')} que imprime \texttt{98}. Con esto es posible preguntar si \texttt{'a'<'b'} lo que es \texttt{True}.

Cree un programa que pida una cadena de letras (en lo posible todas minúsculas o todas mayúsculas) e indique si las letras están ordenadas alfabéticamente (de izquierda a derecha) como el ejemplo siguiente. El programa termina cuando se ingresa el string \texttt{'0'}

\begin{lstlisting}[style=consola]
Ingrese cadena de letras: [*amor*]
El string amor esta ordenado alfabeticamente
Ingrese cadena de letras: [*odio*]
El string odio esta desordenado :(
Ingrese cadena de letras: [*0*]
\end{lstlisting}
