\section{Buses Pythonia}

El glorioso \textit{Imperio de Pythonia} quiere instaurar una empresa de buses para mejorar la calidad de vida de sus habitantes.

Para esto, primero debe crear un sistema de ventas de pasajes, el cual se debe implementar en base a los siguientes requerimientos:

\begin{itemize}
    \item La disponibilidad de asientos sera entregada como una cadena de caracteres (\texttt{string}), el cual contiene las letras \texttt{a} (adulto), \texttt{e} (estudiantes) y \texttt{v} (vacío).

    \item El índice de cada letra de dicha cadena representa cada asiento del bus.
    
    \item Por ejemplo, el string \texttt{vaaaev} indica que el asiento 0 y el asiento 5 están vacíos, el asiento 1, 2 y 3 fueron vendidos a adultos, y el asiento 4 fue vendido a un estudiante.
\end{itemize}

Escriba un programa que permita al usuario ingresar la cadena antes descrita. Luego que pida pasajeros, de los cuales las opciones son tanto \texttt{adulto} como \texttt{estudiante}, y luego el índice del número de asiento a vender. Si el asiento esta ocupado, se debe volver a pedir otro asiento hasta que se ingrese uno desocupado.

El programa debe seguir solicitando datos hasta que se ingrese un ``0'' (cero), o hasta que ya no queden asientos disponibles (asientos vacíos).

Al terminar el programa, este debe mostrar la cantidad total de pasajes comprados, el dinero recaudado y el string actualizado con los datos de los asientos.

Las tarifas son: \$2200 para estudiantes y \$4000 para adultos.

Un detalle importante a considerar es que a la empresa le gusta desarrollar los problemas de forma creativa, por lo que le han prohibido el uso de listas y tuplas.

Por simplicidad, puede considerar que se ingresaran datos validos siempre.

Se adjuntan ejemplos del programa esperado, en los cuales se puede guiar.

\begin{lstlisting}[style=consola]
Ingrese los asientos actuales: [*vaaaev*]

Ingrese tipo de pasaje: [*adulto*]
Ingrese el asiento que quiere comprar: [*0*]
Asignado!

Ingrese tipo de pasaje: [*0*]

Se vendieron 4 pasajes de adulto.
Se vendieron 1 pasajes de estudiante.
Se recaudo 18200
Los asientos finales son: aaaaev
\end{lstlisting}


\begin{lstlisting}[style=consola]
Ingrese los asientos actuales: [*avaaaaaaaaa*]

Ingrese tipo de pasaje: [*estudiante*]
Ingrese el asiento que quiere comprar: [*0*]
Ocupado!
Ingrese el asiento que quiere comprar: [*1*]
Asignado!
Bus lleno.

Se vendieron 10 pasajes de adulto.
Se vendieron 1 pasajes de estudiante.
Se recaudo 42220
Los asientos finales son: aeaaaaaaaaa
\end{lstlisting}
