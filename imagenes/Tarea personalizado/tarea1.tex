\documentclass[spanish, fleqn]{scrartcl}
\usepackage[utf8]{inputenc}
\usepackage{babel}
\usepackage[paper=a4paper, top=2cm, left=2cm, right=2cm]{geometry}
\usepackage{tikz}
\usepackage{CIACcustom}
\usepackage{fourier}
\usepackage{amsmath, amsthm}
\usepackage{listings}
\usepackage{multicol}
\usepackage{fancyhdr}
\usepackage[urlcolor=blue, colorlinks]{hyperref}
\usepackage{booktabs,tabularx}
\usepackage{float}

\newcolumntype{L}[1]{>{\hsize=#1\hsize\raggedright\arraybackslash}X}%
\newcolumntype{R}[1]{>{\hsize=#1\hsize\raggedleft\arraybackslash}X}%
\newcolumntype{C}[2]{>{\hsize=#1\hsize\columncolor{#2}\centering\arraybackslash}X}%

\pagestyle{fancy}
\fancyhf{}
\rhead{\pgfimage[width=2.5cm]{imagenes/logo-ciac.png}}
\chead{
  Apoyos Personalizados Tarea N° 1\\
  IWI-131 Semestre I-2019 \\
  CIAC Casa Central
}
\lhead{\pgfimage[width=2.5cm]{imagenes/logo-usm.jpg}}
\rfoot{\LaTeXe / CIAC 2019}
\lfoot{Página \thepage}
\renewcommand{\headrulewidth}{0.5pt}
\renewcommand{\footrulewidth}{0.5pt}

\renewcommand{\ttdefault}{pcr}

%%% listings settings:
\definecolor{bggray}{rgb}{0.95,0.95,0.95}
\lstdefinestyle{consola}{
  backgroundcolor=\color{bggray},
  basicstyle=\small\ttfamily,
  frame=single,
  moredelim=[is][\bfseries]{[*}{*]},
  xrightmargin=5pt
}

\lstdefinestyle{mypy}{
  language=python,
  backgroundcolor=\color{bggray},
  basicstyle=\ttfamily\small\color{orange!70!black},
  frame=L,
  keywordstyle=\bfseries\color{green!40!black},
  commentstyle=\itshape\color{purple!40!black},
  identifierstyle=\color{blue},
  stringstyle=\color{red},
  numbers=left,
  showstringspaces=false,
  xrightmargin=5pt,
  xleftmargin=10pt
}

\newtheorem{CIACdef}{Definición}

\begin{document}
\vspace*{.3cm}

\section{Signos zodiacales}
Se tiene un diccionario \texttt{fechas} cuya llave es el nombre de un ser viviente y su valor es una tupla de la forma \texttt{anio,mes,dia} que indica la fecha de nacimiento del personaje.

\begin{lstlisting}[style=consola]
fechas={
    'Mike':(1994,7,19),
    'Gary':(1989,5,22),
    'Brad':(1975,5,22),
    'Angie':(1984,5,22),
    'Peter':(1967,12,4),
    'Larry':(2001,3,14),
    'Moe':(2000,12,4)
    #...
    }
\end{lstlisting}

También se tiene una estructura de signos zodiacales

\begin{lstlisting}[style=consola]
signos = {
   'aries':       (( 3, 21), ( 4, 20)),   'tauro':       (( 4, 21), ( 5, 21)),
   'geminis':     (( 5, 22), ( 6, 21)),   'cancer':      (( 6, 22), ( 7, 23)),
   'leo':         (( 7, 24), ( 8, 23)),   'virgo':       (( 8, 24), ( 9, 23)),
   'libra':       (( 9, 24), (10, 23)),   'escorpio':    ((10, 24), (11, 22)),
   'sagitario':   ((11, 23), (12, 21)),   'capricornio': ((12, 22), ( 1, 20)),
   'acuario':     (( 1, 21), ( 2, 19)),   'piscis':      (( 2, 20), ( 3, 20)),
}
\end{lstlisting}
Y una estructura con los elementos de los distintos signos del zodiaco.
\begin{lstlisting}[style=consola]
elementos={
    'tierra':['tauro','virgo','capricornio'],
    'fuego':['aries','leo','sagitario'],
    'aire':['geminis','libra','acuario'],
    'agua':['cancer','escorpion','piscis']
    }
\end{lstlisting}
Se le pide a usted programar las funciones
\begin{itemize}
    \item \texttt{determinar\_signo(fecha)} que recibiendo una tupla con la fecha de forma \texttt{anio,mes,dia} retorne un string con el signo zodiacal correspondiente.
\begin{lstlisting}[style=consola]
>>> determinar_signo((1994,7,19))
'cancer'
>>> determinar_signo((1996,12,27))
'capricornio'
\end{lstlisting}
    \item \texttt{dic\_nombre\_signo(fechas)} que reciba el diccionario de \texttt{fechas} y retorne un diccionario que asocie el nombre de una persona con su signo zodiacal respectivo
\begin{lstlisting}[style=consola]
>>> dic_nombre_signo(fechas)
{'Mike': 'cancer', 'Angie': 'geminis', 'Gary': 'geminis', 'Larry': 'piscis', 
'Brad': 'geminis', 'Peter': 'sagitario', 'Moe': 'sagitario'}
\end{lstlisting}
\newpage
    \item \texttt{elemento(signo)} que reciba un string de un signo zodiacal y retorne el elemento correspondiente (tierra, fuego, aire o agua).
\begin{lstlisting}[style=consola]
>>> elemento('sagitario')
'fuego'
\end{lstlisting}
\end{itemize}

Las funciones anteriores se resuelven con el siguiente código 

 \lstinputlisting[
    style  = mypy,
    caption= \texttt{signos.py}]{signos.py}

\begin{itemize}
    \item \texttt{compatibilidad(fechas)} que reciba el diccionario \texttt{fechas} y retorne un diccionario donde la llave sea un elemento zodiacal y los valores listas que asocien todos los nombres que pertenezcan al elemento correspondiente
\begin{lstlisting}[style=consola]
>>> compatibilidad(fechas)
{'aire': ['Angie', 'Gary', 'Brad'], 'agua': ['Mike', 'Larry'],
'fuego': ['Peter', 'Moe']}
\end{lstlisting}

\end{itemize}
\pagebreak[4]
\section{Abogados}

Pearson Hardman, una firma muy prestigiosa de abogados, requiere ayuda con el manejo de datos de sus empleados. Teniendo el diccionario \texttt{abogados} cuyas llaves son nombres y sus valores otros diccionarios. Estos diccionarios asociados a cada abogado contienen datos de:
\begin{itemize}
    \item Los juicios realizados por mes, bajo la llave \texttt{'juicios'} que asocia una lista de tuplas con la forma \texttt{(mes,cantidad)}
    \item El sueldo que gana por hora el abogado, bajo la llave \texttt{'sueldo'} que asocia un entero
    \item Las empresas que ha defendido el abogado bajo la llave \texttt{'empresas'} que asocia una lista de strings
\end{itemize}

\begin{lstlisting}[style=consola]
abogados={
    'mike':{'juicios':[('julio',3),('agosto',1),('octubre',4)],
            'sueldo':100,
            'empresas':['google','samsung','ciac'] },
    'rachel':{'juicios':[('enero',3),('marzo',4),('julio',8)],
              'sueldo':70,
              'empresas':['google','codelco']},
    'harvey':{'juicios':[('enero',5),('febrero',12),('julio',24)],
              'sueldo':1000,
              'empresas':['mesa verde','samsung']}
    }
\end{lstlisting}

Se le pide a usted crear las siguientes funciones
\begin{itemize}
    \item[a.] \texttt{juicios\_por\_mes(abogados)} que reciba el diccionario \texttt{abogados} y retorne un diccionario que asocie el mes con la cantidad total de juicios realizados.
    \begin{lstlisting}[style=consola]
>>> [*juicios_por_mes(abogados)*]
{'julio': 35, 'marzo': 4, 'agosto': 1, 'enero': 8, 'febrero': 12, 
'octubre': 4}
    \end{lstlisting}
    
    \item[b.] \texttt{total\_juicios(abogado)} que reciba el nombre de un abogado y retorne un entero con la cantidad total de juicios en los que ha estado. Asuma que el diccionario \texttt{abogados} es una variable global del programa.
    \begin{lstlisting}[style=consola]
>>> [*total_juicios('harvey')*]
41
    \end{lstlisting}
    
    \item[c.] \texttt{quien\_trabajo(empresa)} que reciba un string con una empresa y retorne una lista con todos los nombres de los abogados que trabajaron en dicha empresa.
    \begin{lstlisting}[style=consola]
[*>>> quien_trabajo('google')*]
['mike', 'rachel']
    \end{lstlisting}
    \item[d.] \texttt{se\_conocen(abogado\_1,abogado\_2)} que retorne un valor booleano si es que ambos abogados se conocen. Considere que se conocen si es que han trabajado en la misma empresa.
    \begin{lstlisting}[style=consola]
>>> [*se_conocen('mike','rachel')*]
True
>>> [*se_conocen('rachel','harvey')*]
False
    \end{lstlisting}
\end{itemize}


\end{document}