\documentclass[spanish]{scrartcl}
\usepackage[utf8]{inputenc}
\usepackage{babel}
\usepackage[headheight=47pt, paper=a4paper, top=3cm, left=2cm, right=2cm,bottom=3cm]{geometry}
\usepackage{tikz}
\usepackage{CIACcustom}
\usepackage{fourier}
\usepackage{amsmath, amsthm}
\usepackage{listings}
\usepackage{multicol}
\usepackage{fancyhdr}
\usepackage[urlcolor=blue, colorlinks]{hyperref}
\usepackage{booktabs,tabularx}
\usepackage{float}
\usepackage{wrapfig}
\usepackage[normalem]{ulem} %Para tachar palabras y hacer chistes ...

\newcolumntype{L}[1]{>{\hsize=#1\hsize\raggedright\arraybackslash}X}%
\newcolumntype{R}[1]{>{\hsize=#1\hsize\raggedleft\arraybackslash}X}%
\newcolumntype{C}[2]{>{\hsize=#1\hsize\columncolor{#2}\centering\arraybackslash}X}%

%% Cambiar las enumeraciones de numeros a enumeraciones de letras.
\renewcommand{\theenumi}{\Alph{enumi})}


\newcommand{\numCert}{1}
\newcommand{\annoCert}{2019}
\newcommand{\fechaCert}{6 de Abril de \annoCert}

\pagestyle{fancy}
\fancyhf{}
\rhead{\pgfimage[width=2.5cm]{imagenes/logo-ciac.png}}
\chead{
  Simulación Certamen \numCert\\
  IWI-131 Semestre I-\annoCert \\
  CIAC Casa Central
}
\lhead{\pgfimage[width=2.5cm]{imagenes/logo-usm.jpg}}
\rfoot{\LaTeXe / CIAC \annoCert}
\lfoot{\thepage}

\renewcommand{\ttdefault}{pcr}

\renewcommand{\footrulewidth}{0.4pt}% default is 0pt

%%% listings settings:
\definecolor{bggray}{rgb}{0.95,0.95,0.95}
\lstdefinestyle{consola}{
  backgroundcolor=\color{bggray},
  basicstyle=\small\ttfamily,
  frame=single,
  moredelim=[is][\bfseries]{[*}{*]},
  xrightmargin=5pt
}

\lstdefinestyle{mypy}{
  language=python,
  backgroundcolor=\color{bggray},
  basicstyle=\ttfamily\small\color{orange!70!black},
  frame=L,
  keywordstyle=\bfseries\color{green!40!black},
  commentstyle=\itshape\color{purple!40!black},
  identifierstyle=\color{blue},
  stringstyle=\color{red},
  numbers=left,
  showstringspaces=false,
  xrightmargin=5pt,
  xleftmargin=10pt
}

\newtheorem{CIACdef}{Definición}


\begin{document}

\section{Ampolletas}

El maquiavélico archi-enemigo de nuestro noble héroe \textit{Jonh Estrella}, el desalmado sujeto llamado \textit{Brando} ha logrado su objetivo de volverse un vampiro inmortal y súper-poderoso, el cual no pareciera tener debilidades... a excepción de una, la luz solar.

Como este villano es muy inteligente, calculador y precavido, nuestro héroe sabe que no lograra hacer que \textit{Brando} se descuide y salga al exterior de día. Por lo que nuestro héroe, acompañado del veloz e intachable \textit{Wagon}, han decido intentar tenderle una trampa a dicho villano.

Esta trampa consiste en concentrar muchísimas ampolletas, con el objetivo de lograr igualar el poder lumínico del sol, todo esto impulsado por una única batería.

En base a esto, saltan varias inquietudes, como ¿Cuantas ampolletas necesitaran para poder igualar al sol?, ó ¿Cuanta energía necesitarían para tantas ampolletas? Por lo que se ponen a investigar y se dan cuenta de que hay muchas opciones en el mercado, por lo que deben calcular cual es la opción óptima en base a la siguiente ecuación:
\begin{equation*}
    C = I*H*\sqrt[1.15]{\frac{t}{H}}
\end{equation*}

Donde \texttt{C} es Capacidad de la batería [Amperio-Hora], \texttt{I} es Consumo [Amperio], \texttt{H} es Base de tiempo definida por el fabricante [Hora] y \texttt{t} es la Autonomía esperada de la batería [Hora].

Para poder calcular el consumo de las ampolletas, podemos usar la siguiente formula:

\begin{equation*}
    I = \frac{PotenciaAmpolleta_{1}[Watt] + PotenciaAmpolleta_{2}[Watt] + ...}{VoltajeBateria}
\end{equation*}

Sabiendo todo esto, nuestros héroes necesitan poder calcular cual sera la mejor opción a comprar, por lo que le piden a usted hacer un programa en Python que los ayude con tan difícil desafío.

El programa debe pedir el \texttt{H}, la autonomía en horas que quieren que tenga esta batería, el voltaje de la batería, los Watts del sol, y la información de las ampolletas. Todas estas cantidades están en el conjunto de los reales.
Como las ampolletas se venden por packs, donde cada pack tiene un precio, una cantidad de ampolletas y la potencia de cada ampolleta, usted debe ajustarse a eso, pidiendo los datos de cada pack a comprar. Debe dejar de pedir ampolletas cuando la potencia acumulada de todas las ampolletas alcance o supere los Watts del sol. 
Luego debe mostrar por pantalla cual debe ser la capacidad de la batería a comprar y cuanto se gastaría en ampolletas.

A nuestros héroes les gustaría saber que pack tiene la ampolleta mas barata según su potencia y su precio unitario.

Puede guiarse por el siguiente ejemplo de ejecución:

\begin{lstlisting}[style=consola]
Base del tiempo de bateria [Amp-Hr]: [*20*]
Autonomia de bateria [Hr]: [*0.1*]
Voltaje de bateria [Volt]: [*12*]
Potencia del sol [Watt]: [*3000000000000000000*]

Precio pack 1 [$]: [*685937500000000000*]
Cantidad pack 1: 5859375000000000
Potencia pack 1 [Watt]: [*256*]
Precio pack 2 [$]: [*987563750000000000*]
Cantidad pack 2: [*6000000000000000*]
Potencia pack 2 [Watt]: [*1024*]

La capacidad de la bateria a comprar es 1.271375328680958e+17.
Se gastaria 1.67350125e+18 en ampolletas.
El pack mas barato es el 1, con precio unitario 117.06666666666666 por ampolleta
\end{lstlisting}
\pagebreak[4]
\section{X-Espacio}

La reconocida internacionalmente empresa \texttt{X-Espacio}, la cual hace investigaciones espaciales ha encontrado señales que parecieran ser de vida inteligente que intenta comunicarse con nosotros.

Después de un arduo estudio de estas señales, se ha llegado a la conclusión de que estas señales siderales son mensajes codificados en binario, por lo que se le ha pedido a usted, experto programador, que construya un programa capaz de descifrar dichas señales.

Según esta investigación, las ondas corresponden a números binarios desde el 1 al 26, y estos corresponden a una letra de nuestro alfabeto.

Se espera que una secuencia de todos estos caracteres nos entregue frases legibles por nosotros los humanos.

Para transformar un número binario a decimal se debe seguir el siguiente procedimiento:

\begin{itemize}
    \item Se toma el ultimo numero de la secuencia. Debemos llevar un \textit{contador} y una \textit{suma}.
    \item Si este número es "1", le agregamos $2^{contador}$ a la \textit{suma}.
    \item Si el número es "0", lo ignoramos.
    \item Aumentamos en uno al contador.
    \item Repetimos los pasos recién mencionados, pero con el numero siguiente en la secuencia, es decir el que esta a la izquierda del que acabamos de usar.
    \item Nos detenemos cuando ya no queden números por procesar.
\end{itemize}

Tomemos de ejemplo el numero binario \texttt{101110}. El equivalente numérico de dicha secuencia es: $2^{1} + 2^{2} + 2^{3} + 2^{5}$, igual a 46.

Sabiendo esto, su programa debe pedir secuencias binarias hasta que se ingrese un 0, y luego mostrar el mensaje de nuestros amigos espaciales.

Para poder 

\begin{lstlisting}[style=consola]
Ingrese binario: [*101*]
Ingrese binario: [*11000*]
Ingrese binario: [*1001*]
Ingrese binario: [*10100*]
Ingrese binario: [*1111*]
Ingrese binario: [*0*]
exito
\end{lstlisting}
\end{document}