\paragraph{1.} 
[20\%] Una prestigiosa empresa de seguridad computacional, programa algunas funciones para encontrar palabras
dentro de archivos de texto. Lamentablemente, creen que programar borracho es entretenido y eficaz,
por lo que todas sus líneas de código, en un acto de estupidez, fueron mezcladas.

Se tienen las siguientes dos estructuras, \texttt{archivos} una lista que contiene los nombres de los archivos de texto a analizar, y \texttt{malignos} una lista que contiene los strings considerados como palabras que son virus.
\begin{lstlisting}[style=consola]
archivos=['lagrimas','millones','poema 20','poetas']
malignos=['malicia','troyano','mentiroso','dolor']
\end{lstlisting}
Ayude a estos personajes a ordenar su código para que funcione.
\begin{itemize}
    \item La función \texttt{esta\_palabra(archivo,palabra)} que reciba un string palabra y retorne True o False dependiendo si la palabra está o no en archivo.
    
    \begin{lstlisting}[style=consola]
if palabra in linea:
def esta_palabra(archivo,palabra):
for linea in arch:
arch=open(archivo)
return False
arch.close()
return True
arch.close()
    \end{lstlisting}
    
    \item La función \texttt{encontrar\_virus(archivos,malignos)} que reciba una lista de nombres de archivo (sin la extensión .txt) y una lista de strings malignos que contiene palabras consideradas como virus. Esta función retorna una lista con los nombres de los archivos que contienen virus.
    
\begin{lstlisting}[style=consola]
for linea in arch:
arch=open(archivo+'.txt')
arch.close()
def encontrar_virus(archivos,malignos):
return infectados
for palabra in malignos:
infectados.append(archivo+'.txt')
if palabra in linea:
infectados=[]
for archivo in archivos:
if archivo+'.txt' not in infectados:
\end{lstlisting}
\end{itemize}



