\section{Pregunta 2}

Un local de comida rápida quiere digitalizar su sistema de ventas, pero no han podido encontrar ningún sistema pre-hecho que se adapte a sus necesidades, por lo que le han pedido a usted que construya un sistema a medida para ellos.

El mayor atractivo de este local es que el cliente puede combinar cualquier par de productos para hacer un combo. Todas las posibles combinaciones de productos, y sus respectivos precios, las tienen anotadas en una tabla. Usted ha decidido representar esta tabla como una matriz bidimensional de la siguiente forma:

\begin{lstlisting}[style=consola]
productos = [
    "completo italiano", "papas fritas", "chorrillana vegana", 
    "vaso de bebestible", "hamburguesa italiana", "churrasco dinamico",
    "salchipapa", # ...
]
combos = [
    [1200, 2120, 2680, 1320, 2320, 2480, 2280], 
    [2120, 1450, 2880, 1520, 2520, 2680, 2480], 
    [2680, 2880, 2150, 2080, 3080, 3240, 3040], 
    [1320, 1520, 2080,  450, 1720, 1880, 1680], 
    [2320, 2520, 3080, 1720, 1700, 2880, 2680], 
    [2480, 2680, 3240, 1880, 2880, 1900, 2840], 
    [2280, 2480, 3040, 1680, 2680, 2840, 1650], 
]
\end{lstlisting}

La lista \texttt{productos} contiene los nombres de los productos que vende el local.

La matriz \texttt{combos} contiene los precios de los combos posibles. Esta matriz es simétrica por la diagonal. Los índices de esta matriz se corresponden con los índices de la lista \texttt{productos}. Por ejemplo, si queremos comprar el combo de un \textit{baso de bebestible} (índice 3) con una porción de \textit{papas fritas} (índice 1), tenemos que ver el precio almacenado en \texttt{combos[3][1]} (el valor de \texttt{combos[1][3]} también sirve, ya que la matriz es simétrica). La diagonal indica el valor de un único producto (no se puede hacer un combo con el mismo producto 2 veces).

Además, debe almacenar el historial de las ventas realizadas en la siguiente estructura:

\begin{lstlisting}[style=consola]
ventas = [
    # (producto1, producto2, fecha), 
    ("salchipapa", "vaso de bebestible", (2020, 3, 2)), 
    ("chorrillana vegana", "chorrillana vegana", (2020, 3, 2)), 
    ("papas fritas", "completo italiano", (2020, 3, 3)), 
    # ...
]
\end{lstlisting}

En base a esta información, escriba un programa que posea las siguientes opciones:
\begin{itemize}
    \item Ingresar nuevas ventas (preguntando por combos y la fecha de la venta).
    \item Solicitar las ganancias del día especificado.
    \item Agregar un nuevo producto a la lista de \texttt{productos} y su precio a \texttt{combos}, generando los precios de los nuevos posibles combos (puede pedir estos precios por pantalla o generarlos usando la forma que usted estime conveniente, como solicitando un porcentaje de descuento).
    \item Al realizar alguna de las opciones anteriores, el programa debe volver a preguntar que debe hacer. No se debe cerrar hasta que el usuario ingrese la opción de cierre.
\end{itemize} 

