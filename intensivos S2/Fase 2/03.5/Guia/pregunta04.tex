\subsection*{validarMovimientoPeon}

Cree la función \textbf{validarMovimientoPeon(tablero,actual,objetivo,pieza)}, que recibe una lista con la misma estructura que la lista de listas de string \textbf{tablero} del ruteo, una tupla de dos enteros \textbf{actual}, una tupla de enteros \textbf{objetivo}, y un string de una sola letra \textbf{pieza}, y retorna \textbf{True} si esta pieza es un peón y puede moverse a la posición indicada, en todo otro caso retorna \textbf{False}.

\begin{lstlisting}[style=consola]
tablero = [
    ["t","a"," ","a","t"],
    ["p"," ","p","p","p"],
    [" "," "," "," "," "],
    [" ","p","P","P","P"],
    ["T","A"," ","A","T"]]
\end{lstlisting}


La función debe cuidar los siguientes aspectos.
\begin{enumerate}
    \item El peón puede moverse a la casilla que está adelante de él solo si esta está vacía.
    \item El peón puede moverse a la casilla que está adelante y en diagonal de él solo si esta contiene una pieza enemiga.
    \item la dirección de adelante es distinta para el peón del jugador 1 que la del peón del jugador 2
\end{enumerate}

%\begin{outcode}{CodigosImportados/Ajedrez/validarPeon.txt}{0}{100}
%\end{outcode}


\begin{lstlisting}[style=consola]
>>> [*print(validarMovimientoPeon(tablero, (1,2), (0,2), 'p'))*]
False
>>> [*print(validarMovimientoPeon(tablero, (1,2), (2,2), 'P'))*]
True
>>> [*print(validarMovimientoPeon(tablero, (3,2), (2,2), 'P'))*]
True
>>> [*print(validarMovimientoPeon(tablero, (3,2), (2,1), 'P'))*]
False
>>> [*print(validarMovimientoPeon(tablero, (3,1), (4,0), 'p'))*]
True
>>> [*print(validarMovimientoPeon(tablero, (3,1), (4,2), 'p'))*]
False
\end{lstlisting}


