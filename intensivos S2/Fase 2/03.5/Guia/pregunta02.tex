\subsection*{validarMovimientoTorre}

Cree la función \texttt{validarMovimientoCaballo(tablero, actual, objetivo, pieza)}, que recibe una lista con la misma estructura que la lista de listas de string \texttt{tablero} del ruteo, una tupla de dos enteros \texttt{actual}, una tupla de enteros \texttt{objetivo}, y un string de una sola letra \texttt{pieza}. Esta función retorna \texttt{True} si esta pieza es una torre y puede moverse a la posición indicada, en todo otro caso retorna \texttt{False}.

\begin{lstlisting}[style=consola]
tablero = [
    ["t","a","c","a","t"],
    ["p"," ","p","p","p"],
    [" "," "," "," "," "],
    [" ","p","P","P","P"],
    ["T","A","C","A","T"]]
\end{lstlisting}

La función debe cuidar los siguientes aspectos.
\begin{enumerate}
    \item La torre solo puede moverse de manera horizontal o vertical.
    \item La torre no puede saltarse otra pieza, por lo que debe verificar que no haya ninguna pieza interponiéndose entre esta y su destino.
    \item La torre puede caer sobre una pieza enemiga.
    \item La torre no puede caer sobre una pieza aliada.
\end{enumerate}

%\begin{outcode}{CodigosImportados/Ajedrez/validarTorre.txt}{0}{100}
%\end{outcode}

\begin{lstlisting}[style=consola]
>>> [*print(validarMovimientoTorre(tablero, (0,0), (2,0), 't'))*]
False
>>> [*print(validarMovimientoTorre(tablero, (0,0), (1,1), 't'))*]
False
>>> [*print(validarMovimientoTorre(tablero, (4,0), (2,0), 'T'))*]
True
\end{lstlisting}
