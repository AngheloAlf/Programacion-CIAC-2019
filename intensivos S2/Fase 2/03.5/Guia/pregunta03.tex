\subsection*{validarMovimientoCaballo}

Cree la función \textbf{validarMovimientoCaballo(tablero,actual,objetivo,pieza)}, que recibe una lista con la misma estructura que la lista de listas de string \textbf{tablero} del ruteo, una tupla de dos enteros \textbf{actual}, una tupla de enteros \textbf{objetivo}, y un string de una sola letra \textbf{pieza}. Esta función retorna \textbf{True} si esta pieza es un caballo y puede moverse a la posición indicada, en todo otro caso retorna \textbf{False}. \\
La función debe cuidar los siguientes aspectos.
\begin{enumerate}
    \item El caballo solo puede moverse a una casilla cuya distancia esté a $\sqrt{5}$ espacios.\\
    La formula para calcular la distancia entre dos casillas es $\sqrt{(x_a-x_b)^2+(y_a-y_b)^2}$.
    \item El caballo puede saltar sobre otras piezas.
    \item El caballo puede caer sobre una pieza enemiga.
    \item El caballo no puede caer sobre una pieza aliada.
\end{enumerate}

%\begin{outcode}{CodigosImportados/Ajedrez/validarCaballo.txt}{0}{100}
%\end{outcode}


\begin{lstlisting}[style=consola]
>>> [*print(validarMovimientoCaballo(tablero, (0,2), (2,3), 'c'))*]
True
>>> [*print(validarMovimientoCaballo(tablero, (0,2), (2,2), 'c'))*]
False
>>> [*print(validarMovimientoCaballo(tablero, (0,2), (1,4), 'c'))*]
False
\end{lstlisting}
