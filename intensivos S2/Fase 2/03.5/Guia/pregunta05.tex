\subsection*{calcularPuntajeJugadores}

Cree la función \textbf{calcularPuntajeJugadores(tablero,valores)}, que recibe una lista de listas de String  \textbf{tablero}, que contiene strings espacio o strings de una sola letra en mayúscula (piezas del jugador 1) o minúscula(piezas del jugador 2), y un diccionario valores cuyas llaves son un string en mayúscula correspondiente a una pieza y su valor asociado es el valor de la pieza.

\begin{lstlisting}[style=consola]
tablero = [
    ["t","a"," "," ","t"],
    ["p"," ","p"," ","p"],
    [" "," "," "," "," "],
    [" ","p","P"," ","P"],
    ["T"," "," "," ","T"]]

valores = {
    "P":1,
    "T":5,
    "C":3,
    "A":3}
\end{lstlisting}


Esta función debe imprimir por pantalla una frase que informe el puntaje total que tiene cada jugador. Este puntaje corresponde a la suma de los puntajes individuales de cada pieza de cada jugador y retornar una tupla con los mismos puntajes del jugador1 y del jugador2 en ese orden.

\begin{lstlisting}[style=consola]
>>> [*print(calcularPuntajeJugadores(tablero,valores))*]
El jugador 1 tiene 12 puntos y el jugador 2 tiene 17
(12,17)
\end{lstlisting}
