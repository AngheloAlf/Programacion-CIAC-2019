\section*{Cine casero}

Pasados 30 años desde ahora, usted, un connotado, renombrado y millonario ingeniero ha construido un cine en su casa tal como se lo recomendó su buen amigo, un tutor de programación. Para la venta de entradas quiere programar un robot que haga el trabajo de un cajero.


En el cine existen 20 butacas, y las funciones siempre comienzan cuando se han vendido todas las entradas. Existen dos precios dependiendo la naturaleza del comprador, estos son 

\begin{itemize}
    \item Niño: que paga \$ 1000
    \item Adulto: que paga \$ 3000
\end{itemize}

Escriba un código en Python para programar al robot vendedor de entradas. Este debe preguntar primero por la cantidad de entradas de niños y luego por la cantidad de entradas para adultos que quiere comprar. 

Cuando la cantidad de entradas exceda los asientos disponibles del cine, debe imprimir por pantalla que no quedan tantos boletos, por el contrario, cuando la venta puede ser realizada, debe imprimir el precio a pagar. 

Al ser vendidas todas las entradas, el robot debe imprimir las ganancias del día.

\begin{lstlisting}[style=consola]
Ingrese entradas ninio: [*6*]
Ingrese entradas adulto: [*10*]
Debe pagar $36000
Ingrese entradas ninio: [*3*]
Ingrese entradas adulto: [*2*]
No quedan tantos boletos!
Ingrese entradas ninio: [*3*]
Ingrese entradas adulto: [*1*]
Debe pagar $6000

Todas las entradas vendidas :D
Hoy ganaste $ 42000
\end{lstlisting}