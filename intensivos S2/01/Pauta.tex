\documentclass[spanish, fleqn]{scrartcl}
\usepackage[utf8]{inputenc}
\usepackage{babel}
\usepackage[paper=a4paper, top=2cm, left=2cm, right=2cm]{geometry}
\usepackage{tikz}
%\usepackage{CIACcustom}
\usepackage{fourier}
\usepackage{amsmath, amsthm}
\usepackage{listings}
\usepackage{multicol}
\usepackage{fancyhdr}
\usepackage[urlcolor=blue, colorlinks]{hyperref}
\usepackage{booktabs,tabularx}
\usepackage{float}

\newcolumntype{L}[1]{>{\hsize=#1\hsize\raggedright\arraybackslash}X}%
\newcolumntype{R}[1]{>{\hsize=#1\hsize\raggedleft\arraybackslash}X}%
\newcolumntype{C}[2]{>{\hsize=#1\hsize\columncolor{#2}\centering\arraybackslash}X}%

\renewcommand{\lstlistingname}{Código}

\pagestyle{fancy}
\fancyhf{}
\rhead{\pgfimage[width=2.5cm]{imagenes/logo-ciac.png}}
\chead{
  Apoyos Intensivos Pauta N°1\\
  IWI-131 Semestre II-2019 \\
  CIAC Casa Central
}
\lhead{\pgfimage[width=2.5cm]{imagenes/logo-usm.jpg}}
\rfoot{\LaTeXe / CIAC 2019}
\lfoot{\thepage}

\renewcommand{\ttdefault}{pcr}

%%% listings settings:
\definecolor{bggray}{rgb}{0.95,0.95,0.95}
\lstdefinestyle{consola}{
  backgroundcolor=\color{bggray},
  basicstyle=\small\ttfamily,
  frame=single,
  moredelim=[is][\bfseries]{[*}{*]},
  xrightmargin=5pt
}

\lstdefinestyle{mypy}{
  language=python,
  backgroundcolor=\color{bggray},
  basicstyle=\ttfamily\small\color{orange!70!black},
  frame=L,
  keywordstyle=\bfseries\color{green!40!black},
  commentstyle=\itshape\color{purple!40!black},
  identifierstyle=\color{blue},
  stringstyle=\color{red},
  numbers=left,
  showstringspaces=false,
  xrightmargin=5pt,
  xleftmargin=10pt
}

\newtheorem{CIACdef}{Definición}

\begin{document}
\vspace*{.3cm}
\section{Valores booleanos}

En orden, los valores que imprime Python son

\begin{itemize}
    \item \texttt{True}
    \item \texttt{False}
    \item \texttt{True}
    \item \texttt{False}
    \item \texttt{False}
    \item \texttt{False}
    \item \texttt{False}
    \item \texttt{True}
\end{itemize}

\section{Cálculo de $\pi$}

\subsection{Ahora un poco más complicado}

\begin{lstlisting}[style=consola]
suma=0
i=0
while i<=155:
	suma+= i**4 *3*i
	i+=1
print(suma)
\end{lstlisting}

\subsection{Ahora si que si}

\lstinputlisting[style=mypy, caption=\texttt{pi.py}]{Code/pi.py}
\newpage
\section{Cine casero}

\lstinputlisting[style= mypy,
                caption=\texttt{cine.py}]{Code/p01.py}

\end{document}