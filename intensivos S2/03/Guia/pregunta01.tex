\section{Impuesto a la renta}

Para practicar su buena creación de funciones, considere que existe un programa para calcular los impuestos generados por varias entidades, ya sean empresas o personas. Con respecto a lo anterior los impuestos se calculan de la siguiente manera
\begin{itemize}
    \item En caso de ser una empresa, el impuesto es del 27\%
    \item En caso de ser una persona, el impuesto varía según su renta
    \begin{table}[H]
\centering
\begin{tabular}{|c|c|c|}
\hline
\multicolumn{2}{|c|}{\textbf{Renta líquida}} & \textbf{Impuesto} \\ \hline
\textbf{Desde} & \textbf{Hasta} & \textbf{Porcentaje} \\ \hline
- & 664.591 & 0 \\ \hline
664.592 & 1.476.870 & 4 \\ \hline
1.476.871 & 2.461.450 & 8 \\ \hline
2.461.451 & 3.446.030 & 13,5 \\ \hline
3.446.031 & 4.430.610 & 23 \\ \hline
4.430.611 & 5.907.480 & 30,4 \\ \hline
5.907.481 & Y MAS & 35 \\ \hline
\end{tabular}
\end{table}
\end{itemize}

El programa tiene una estructura como la siguiente, pero falta completar la información que está tapada con signos de pregunta (?). Complete el algoritmo y cree la función \texttt{impuesto(monto,tipo)} que reciba un entero \texttt{monto} con la renta líquida de la entidad, y un string \texttt{tipo} con el tipo de entidad (\textit{empresa} o \textit{persona}).

\lstinputlisting[style=mypy,caption=\texttt{impuesto.py}]{Code/vacio.py}

El funcionamiento del programa anterior debería ser

\begin{lstlisting}[style=consola]
Ingrese tipo de entidad: [*empresa*]
Ingrese renta liquida: [*3000000*]
Ingrese tipo de entidad: [*persona*]
Ingrese renta liquida: [*3580000*]
Ingrese tipo de entidad: [*persona*]
Ingrese renta liquida: [*300000*]
Ingrese tipo de entidad: # Aqui se presiona enter sin ingresar nada 
El total recaudado fue 1633400.0
\end{lstlisting}