\section{Gualkin Ded (C2 2014-2 CCyCSSJ)}

El mundo esta sumido en el caos. Se desató una epidemia, la cual ha arrasado con la mayor parte de la población. Los humanos infectados vuelven a la vida como “walkers”, muertos vivientes hambrientos de carne humana. Sin embargo, estos walkers no son el único peligro, sino que muchas veces los humanos deben matar a otros para sobrevivir. En este mundo post apocalíptico es su misión desarrollar una aplicación en Python para clasificar a los miembros de su grupo de sobrevivientes. Suponga que las estadísticas de grupo se encuentran en el siguiente
diccionario, cuya clave es el nombre del miembro del grupo y valor una tupla con la cantidad de walkers y la cantidad de humanos eliminados (ej: rick ha matado 172 walkers y 10 humanos):

\begin{lstlisting}[style=consola]
grupo = {
    'rick':(172,10), 'daryl':(136,11), 'michonne':(80,6),
    'glenn':(73,0), 'maggie':(55,4), 'carl':(62,1),
    'tyreese':(35,0), 'carol':(17,3) }
\end{lstlisting}

Obviamente este diccionario es solo un ejemplo, ya que probablemente mas de algún miembro morirá mientras usted desarrolla esta aplicación. 

\begin{itemize}
    \item[a)] Desarrolle la función \texttt{total(grupo)} que recibe como parámetro un diccionario con las estadísticas del grupo, retornando como una tupla el total de walkers y el total de humanos eliminados.
    \begin{lstlisting}[style=consola]
>>> [*total(grupo)*]
(630.0, 35.0)
    \end{lstlisting}
    \item[b)] Desarrolle la función \texttt{puntaje(grupo)} que recibe como parámetro un diccionario con las estadísticas del grupo, y retorna un diccionario cuya clave es el nombre del miembro del grupo y valor el puntaje asociado. Este puntaje se calcula como: (walkers/total walkers) + 2 ∗ (humanos/total humanos), en donde walkers representa a los walkers eliminados por el miembro del grupo, y el total walkers representa al total de walkers eliminado por todo el grupo. humanos y total humanos sigue la misma lógica. Redondee a la centésima.
    \begin{lstlisting}[style=consola]
>>> [*puntaje(grupo)*]
{'maggie': 0.32, 'glenn': 0.12, 'michonne': 0.47, 'rick': 0.84, 'carl': 0.16, 
'carol': 0.2, 'daryl': 0.84, 'tyreese': 0.06}
    \end{lstlisting}
    \item[c)]Desarrolle la función \texttt{ranking(grupo)} que recibe como parámetro un diccionario con las estadísticas del grupo, y retorna una lista con los nombres de los miembros del grupo ordenados de \textbt{mayor a menor} (según el puntaje definido en la parte b).
    \begin{lstlisting}[style=consola]
>>> [*ranking(grupo)*]
['rick','daryl', 'michonne','maggie', 
'carol', 'carl', 'glenn', 'tyreese']
    \end{lstlisting}
\end{itemize}