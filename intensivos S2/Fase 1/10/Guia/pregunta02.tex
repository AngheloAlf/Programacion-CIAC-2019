\section{¡Tenemos que ponerle mas cosas!}

Hemos mostrado nuestro avance del prototipo del nuevo servidor de \textbf{NOMBRE}, y les ha encantado a los dueños, por lo que le han pedido que siga agregando funcionalidades.

Le han entregado un extracto del archivo que contiene los comentarios de la página:
\begin{center}
    \texttt{comentarios.csv} \\
	\begin{tabular}{|l|}
		\hline
comentario\_id,post\_id,usuario\_id,contenido,upvotes,downvotes,fecha\\
1,39,43,jajajajajajaja,83,9,2013-06-14\\
17,42,3,pucha; le crei,483,13,2008-07-24\\
37,42,43,Gracias a este video pude sacar mi titulo <3,483,13,2012-12-01\\
5,39,3,oh que penita :(,115,7,2013-07-11\\
29,34,24,Esta cancion me ayuda caleta para estudiar progra,25,1,2019-04-17\\
24,42,666,100\% real no fake,7483,347,2009-10-25\\
3,53,666,que adorable,425,24,2019-05-15\\
47,47,24,esto es la motivacion de mi vida,987,4,2020-01-04\\
		\hline
	\end{tabular}
\end{center}

Nuevamente, la primera linea indica los nombres de las columnas.

\begin{itemize}
\itemsep0em 
    \item \texttt{comentario\_id} es el identificador único de dicho comentario, es un valor numérico.
    \item \texttt{post\_id} es el identificador único del post al cual corresponde este comentario, es un valor numérico. Ojo que un post puede tener múltiples comentarios.
    \item \texttt{usuario\_id} es el identificador único de usuario que ha escrito este comentario, es un valor numérico.
    \item \texttt{contenido} es el comentario.
    \item \texttt{upvotes} representa la cantidad de votos positivos que ha recibido este comentario.
    \item \texttt{downvotes} representa la cantidad de votos negativos que ha recibido este comentario.
    \item \texttt{fecha} es la fecha en que se realizó este comentario. Tiene el formato \texttt{AÑO-MES-DIA}.
\end{itemize}

Considerando este nuevo archivo, se ha planteado crear las siguientes funcionalidades:
\begin{enumerate}

    \item[$\frac{d}{dx}$. ] \texttt{comentar(archComs, archPosts, archUsers, idPost, username, com, fecha)}. Esta función recibe los 3 archivos antes descritos, el \textit{id} (como \textit{string}) de un post a comentar, el nombre de usuario de quien está comentando (\texttt{username}), el comentario (\texttt{com}) que va a comentar, y una \textit{tupla} de 3 \textit{enteros} indicando la \texttt{fecha}.

    Se debe agregar el comentario al final del archivo de comentarios, generando un \textit{id} valido y cambiando todas las comas ('\texttt{,}') por punto y comas ('\texttt{;}'). Además, se debe verificar que el usuario y el post existan, si no existen, el comentario no debe ser añadido.

    Esta función retorna \texttt{True} si se pudo comentar, o \texttt{False} si no existía el post o el usuario.

    \begin{lstlisting}[style=consola]
>>> [*com = "que adorable!"*]
>>> [*fecha = (2020, 2, 12)*]
>>> [*comentar("comentarios.csv", "posts.csv", "usuarios.csv", "1",
"ultimate_chef", com, fecha)*]
True
    \end{lstlisting}

\end{enumerate}

\begin{center}
    \texttt{comentarios.csv} \\
	\begin{tabular}{|l|}
		\hline
comentario\_id,post\_id,usuario\_id,contenido,upvotes,downvotes,fecha\\
\# lineas omitidas... \\
47,28,24,esto es la motivacion de mi vida,987,4,2020-01-04\\
48,1,43,que adorable!,0,0,2020-2-12\\
		\hline
	\end{tabular}

Se omitieron la mayoría de las lineas por comodidad, el archivo real debería tenerlas.
\end{center}

\begin{enumerate}

    \item[$\sum$. ] \texttt{comentariosDePost(archComs, archPosts, idPost)}. Recibe el nombre del archivo de comentarios, el archivo de posts, y el \textit{id} de un post. 

    Retorna un diccionario, donde las llaves son el \textit{id} del comentario, y los valores son diccionarios. Cada uno de estos sub-diccionarios tienen como llaves las columnas de el archivo de comentarios, y como valores tiene los respectivos valores de cada fila que corresponden a los comentarios del post pedido. Estos valores deben ser convertidos a los tipos de datos correspondientes (\textit{int}s, \textit{listas}, etc).

    El diccionario que retorna esta función debe contener solo los comentarios asociados al \texttt{idPost}.

    Si el post no tiene comentarios, esta función retorna un diccionario vacío.
    
    Si no existe un post con el \textit{id} \texttt{idPost}, esta función retorna \texttt{None}.

    \begin{lstlisting}[style=consola]
>>> [*print(comentariosDePost("comentarios.csv", "posts.csv", "39"))*]
{1: {'comentario_id': 1, 'post_id': 39, 'upvotes': 83, 'downvotes': 9, 
'contenido': 'jajajajajajaja', 'fecha': (2013, 6, 14), 'usuario_id': 43}, 
5: {'comentario_id': 5, 'post_id': 39, 'upvotes': 115, 'downvotes': 7, 
'contenido': 'oh que penita :(', 'fecha': (2013, 7, 11), 'usuario_id': 3}}
>>> [*print(comentariosDePost("comentarios.csv", "posts.csv", "45"))*]
{}
>>> [*print(comentariosDePost("comentarios.csv", "posts.csv", "800"))*]
None
    \end{lstlisting}

    \item [$\int$. ] \texttt{borrarComentariosFechaErronea(archComs, archPosts, archUsers)}. Recibe el nombre de los archivos antes explicados. Debe buscar y eliminar los comentarios que tengan una fecha errónea.
    
    Los comentarios con fechas erróneas son aquellos comentarios que tienen una fecha anterior a la fecha del post, o que su fecha es anterior a la fecha de creación del usuario que comentó.

    Retorna una lista con los \textit{id}s de los comentarios eliminados.

    \begin{lstlisting}[style=consola]
>>> [*print(borrarComentariosFechaErronea("comentarios.csv", "posts.csv", "usuarios.csv"))*]
['1', '17']
    \end{lstlisting}

\end{enumerate}

\begin{center}
    \texttt{comentarios.csv} \\
	\begin{tabular}{|l|}
		\hline
comentario\_id,post\_id,usuario\_id,contenido,upvotes,downvotes,fecha\\
37,42,43,Gracias a este video pude sacar mi titulo <3,483,13,2012-12-01\\
5,39,3,oh que penita :(,115,7,2013-07-11\\
29,34,24,Esta cancion me ayuda caleta para estudiar progra,25,1,2019-04-17\\
24,42,666,100\% real no fake,7483,347,2009-10-25\\
3,53,666,que adorable,425,24,2019-05-15\\
47,47,24,esto es la motivacion de mi vida,987,4,2020-01-04\\
48,1,43,que adorable!,0,0,2020-2-12\\
		\hline
	\end{tabular}

\end{center}

\begin{enumerate}

    \item[$\star$. ] \texttt{eliminarPost(archPosts, archComs, idPost)}. Esta función recibe el nombre del archivo de posts ( \texttt{archPosts}), el nombre del archivo de comentarios (\texttt{archComentarios}) y el \textit{id} de un post a eliminar (\texttt{idPost}).
    
    Esta función debe eliminar el post asociado a \texttt{idPost}, y eliminar los comentarios asociados a dicho post.

    Está función retorna la cantidad de comentarios eliminados.
,  
    \begin{lstlisting}[style=consola]
>>> [*print(eliminarPost("posts.csv", "comentarios.csv", "42"))*]
2
    \end{lstlisting}
    

    \item[$\heartsuit$.] \texttt{actualizarPuntajeUsers(archUsers, archPosts, archComs)}. Esta función recibe el nombre de los 3 archivos anteriormente descritos. 

    Esta función debe buscar todos los posts y comentarios que ha realizado un usuario en especifico, sumar sus \texttt{upvotes} y restar sus \texttt{downvotes}, y este resultado debe ser actualizado en la columna \texttt{puntaje} del archivo de usuarios en el usuario correspondiente.

    Esta función retorna 2 valores, la suma de todos los \texttt{upvotes} y la suma de todos los \texttt{downvotes}.

    \begin{lstlisting}[style=consola]
>>> [*print(actualizarPuntajeUsers("usuarios.csv", "posts.csv", "comentarios.csv"))*]
(83341, 2628)
    \end{lstlisting}

\end{enumerate}

\begin{center}
    \texttt{usuarios.csv} \\
	\begin{tabular}{|l|}
		\hline
user\_id,nombre,fecha\_creacion,puntaje,cantidad\_posts\\
3,animal\_lover,2007-09-23,6109,3\\
24,jaja xd,2017-11-11,11114,1\\
42,cool\_guy,2012-02-14,54158,5\\
43,ultimate\_chef,2008-04-13,9303,1\\
666,jesucritos,0020-07-25,29,2\\
667,full detonao,2020-2-12,0,0\\
		\hline
	\end{tabular}
\end{center}
