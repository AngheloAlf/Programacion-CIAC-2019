\section{Signos zodiacales}
Se tiene un diccionario \texttt{fechas} cuya llave es el nombre de un ser viviente y su valor es una tupla de la forma \texttt{anio,mes,dia} que indica la fecha de nacimiento del personaje.

\begin{lstlisting}[style=consola]
fechas={
    'Mike':(1994,7,19),
    'Gary':(1989,5,22),
    'Brad':(1975,5,22),
    'Angie':(1984,5,22),
    'Peter':(1967,12,4),
    'Larry':(2001,3,14),
    'Moe':(2000,12,4) }
\end{lstlisting}

También se tiene una estructura de signos zodiacales

\begin{lstlisting}[style=consola]
signos = {
   'aries':       (( 3, 21), ( 4, 20)),   'tauro':       (( 4, 21), ( 5, 21)),
   'geminis':     (( 5, 22), ( 6, 21)),   'cancer':      (( 6, 22), ( 7, 23)),
   'leo':         (( 7, 24), ( 8, 23)),   'virgo':       (( 8, 24), ( 9, 23)),
   'libra':       (( 9, 24), (10, 23)),   'escorpio':    ((10, 24), (11, 22)),
   'sagitario':   ((11, 23), (12, 21)),   'capricornio': ((12, 22), ( 1, 20)),
   'acuario':     (( 1, 21), ( 2, 19)),   'piscis':      (( 2, 20), ( 3, 20))}
\end{lstlisting}
Y una estructura con los elementos de los distintos signos del zodiaco.
\begin{lstlisting}[style=consola]
elementos={
    'tierra':['tauro','virgo','capricornio'],
    'fuego':['aries','leo','sagitario'],
    'aire':['geminis','libra','acuario'],
    'agua':['cancer','escorpion','piscis']    }
\end{lstlisting}
Se le pide a usted programar las funciones
\begin{itemize}
    \item \texttt{determinar\_signo(fecha)} que recibiendo una tupla con la fecha de forma \texttt{anio,mes,dia} retorne un string con el signo zodiacal correspondiente.
\begin{lstlisting}[style=consola]
>>> [*determinar_signo((1994,7,19))*]
'cancer'
>>> [*determinar_signo((1996,12,27))*]
'capricornio'
\end{lstlisting}
    \item \texttt{dic\_nombre\_signo(fechas)} que reciba el diccionario de \texttt{fechas} y retorne un diccionario que asocie el nombre de una persona con su signo zodiacal respectivo
\begin{lstlisting}[style=consola]
>>> [*dic_nombre_signo(fechas)*]
{'Mike': 'cancer', 'Angie': 'geminis', 'Gary': 'geminis', 'Larry': 'piscis', 
'Brad': 'geminis', 'Peter': 'sagitario', 'Moe': 'sagitario'}
\end{lstlisting}
    \item \texttt{elemento(signo)} que reciba un string de un signo zodiacal y retorne el elemento correspondiente (tierra, fuego, aire o agua).
\begin{lstlisting}[style=consola]
>>> [*elemento('sagitario')*]
'fuego'
\end{lstlisting}
    \item \texttt{compatibilidad(fechas)} que reciba el diccionario \texttt{fechas} y retorne un diccionario donde la llave sea un elemento zodiacal y los valores listas que asocien todos los nombres que pertenezcan al elemento correspondiente
\begin{lstlisting}[style=consola]
>>> [*compatibilidad(fechas)*]
{'aire': ['Angie', 'Gary', 'Brad'], 'agua': ['Mike', 'Larry'],
'fuego': ['Peter', 'Moe']}
\end{lstlisting}

\end{itemize}
