\section{El Juego de la Vida de Conway}

El juego de la vida de Conway se trata de un juego de cero jugadores donde se tiene una matriz de células cuyo estado puede ser viva o muerta. El juego inicia con una matriz de células ya creadas, cada célula decide si morirá o vivirá dependiendo del total de células vivas que tiene en su vecindad. A continuación se detallan algunas reglas y definiciones:

\begin{itemize}
    \item La vecindad de una célula $c$ es el conjunto de células vecinas conformado por la célula que esta arriba, abajo, a la izquierda, la derecha y las esquinas de la célula $x$. Es decir si $c_{i,j}$ representa una célula en la posición $(i,j)$, sus vecinos son $V(i,j)=\{c_{i-1,j-1}, c_{i,j-1}, c_{i+1,j-1}, c_{i-1,j}, c_{i+1,j}, c_{i-1,j+1}, c_{i,j+1}, c_{i+1,j+1}\}$
    
    \item La supervivencia de cada célula se determina con las siguientes reglas:
    \begin{enumerate}
        \item Si la célula esta viva, continuara viva si tiene 2 o 3 vecinos vivos, de lo contrario muere.
        \item Si la célula esta muerta, estará viva si tiene 3 vecinos, de lo contrario continuará muerta.
    \end{enumerate}
    
    La forma en que se representará la matriz de células es mediante un archivo de texto llamado \texttt{m1.txt} y \texttt{m2.txt}. La idea es leer estos archivos y guardar la matriz inicial de células como una lista bidimensional donde cada elemento de la lista es una columna y cada columna tiene un 0 si la célula esta muerta o un 1 si la célula esta viva. La estructura de cada archivo se detalla a continuación.
    
\end{itemize}
\begin{itemize}
    \item \texttt{m1.txt} es simplemente una matriz de unos y ceros, donde cada linea del archivo representa una fila de la matriz de células.
    
    \begin{center}
        \texttt{m1.txt} \\
    	\begin{tabular}{|l|}
    		\hline
            0,0,0,1,...,0,0\\
            0,0,0,1,...,0,0\\
            0,0,0,1,...,0,0\\
            1,1,1,0,...,0,0\\
            .\\
            .\\
            .\\
            0,0,0,0,...,0,0\\
    		\hline
    	\end{tabular}
    \end{center}
    
    \item \texttt{m2.txt} es un archivo con una serie de instrucciones para construir la matriz de células.
    
    \begin{center}
        \texttt{m2.txt} \\
    	\begin{tabular}{|l|}
    		\hline
            50 50\\
            glider 1,1\\
            t 30 30\\
            r-pentomino 4, 40\\
    		\hline
    	\end{tabular}
    \end{center}
    
    La primera linea de \texttt{m2.txt} detalla las dimensiones separados por un espacio.
    Las demás lineas detalla el nombre de la figura y su posición en la matriz de células.
    Considere que las figuras detalladas en el archivo pueden ser obtenidas mediante el archivo \texttt{patrones.py} incluido.
\end{itemize}

Ahora para poder jugar este juego se pide que haga las siguientes funciones:

\begin{itemize}
\item[a.] Desarrollar la función \texttt{mundo\_muerto(ancho, largo)} que reciba como parámetro el ancho de la matriz de células y el largo de esta. entendiendo como el ancho la cantidad de filas que tiene la matriz y el largo la cantidad de columnas que tiene esta matriz. Como retorno se espera obtener una matriz de células muertas con las dimensiones especificadas.

\begin{lstlisting}[style=consola]
>>> [*mundo_muerto(2, 3)*]
[[0,0],[0,0],[0,0]]
\end{lstlisting}

\item[b.] Desarrollar la función \texttt{vecinos\_vivos(mundo, x, y)} que reciba como parámetro la matriz de células, y la posición $x$, $y$ de la célula que se quiere analizar sus vecinos. Como resultado se debe retornar el numero total de vecinos vecinos vivos.
\begin{lstlisting}[style=consola]
>>> mundo = [[1,0,0],[1,1,1],[0,0,0]]
>>> [*vecinos_vivos(mundo, 1, 1)*]
3
>>> [*vecinos_vivos(mundo, 0, 0)*]
2
\end{lstlisting}

\item[c.] Programar la función \texttt{sig\_estado(mundo)} que reciba como parámetro la matriz de células, y retorne una nueva matriz de células producto de la aplicación de las reglas de supervivencia.
\begin{lstlisting}[style=consola]
>>> mundo = [[0,0,0],[1,1,1],[0,0,0]]
>>> [*sig_estado(mundo)*]
[[0,1,0],[0,1,0],[0,1,0]
\end{lstlisting}

\item[d.] Crear la función \texttt{leer\_mundo\_por\_comando(nombre\_mundo)} que reciba como parámetro un string con el nombre del archivo a abrir y que sea la estructura del archivo \texttt{m2.txt} y retorne una lista bidimensional con cada elemento de la lista como una columna de la matriz de células construida por las instrucciones.

\begin{lstlisting}[style=consola]
>>> [*leer_mundo_por_comando(nombre_mundo)*]
[[0,1,...,0],[0,1,...,0],...,[0,0,...,0]
\end{lstlisting}

\item[e.] Crear la función \texttt{leer\_mundo(nombre\_mundo)} que reciba como parámetro un string con el nombre del archivo a abrir y que sea la estructura del archivo \texttt{m1.txt} y retorne una lista bidimensional con cada elemento de la lista como una columna de la matriz de células.

\begin{lstlisting}[style=consola]
>>> [*leer_mundo_por_comando(nombre_mundo)*]
[[0,1,...,0],[0,1,...,0],...,[0,0,...,0]
\end{lstlisting}

\item[f.] Crear la función \texttt{dibujar(arch, mundo)} que reciba como parámetro el archivo de texto abierto y la matriz de células como mundo a dibujar en el archivo

\begin{lstlisting}[style=consola]
>>> arch = open("test.txt", "w", encoding="utf8")
>>> mundo = [[0,1,0],[0,1,0],[0,1,0]
>>> [*dibujar(arch, mundo)*]
>>> arch.close()
\end{lstlisting}

El archivo de texto resultante es:

\begin{center}
    \texttt{test.txt} \\
	\begin{tabular}{|l|}
		\hline
        $\blacksquare \blacksquare \blacksquare$\\
        $\square \square \square$\\
        $\blacksquare \blacksquare \blacksquare$\\
		\hline
	\end{tabular}
\end{center}

\item[g.] Crear la función \texttt{simular(max\_generacion, lectura, nombre\_mundo)} que reciba como parámetro un entero indicando la cantidad de generaciones a simular, el modo de lectura del archivo, y nombre del archivo a leer. Como resultado la función debe crear un archivo txt con el nombre del archivo de lectura concatenado con "simulacion\_CGol.txt", con los dibujos de cada generación y con un numero que indique la cantidad de células vivas en la generación.

\begin{lstlisting}[style=consola]
>>> [*simular(3, "comando", "mtest.txt")*]
\end{lstlisting}
\end{itemize}

\begin{center}
\begin{tabular}{c c}
    \texttt{mtest.txt} & \texttt{mtest\_simulacion\_CGol.txt} \\
	\begin{tabular}{|l|}
		\hline
        3 3\\
        blinker 1,0\\
		\hline
	\end{tabular} & \begin{tabular}{|l|}
	    \hline
        GENERACION:0   POBLACION:3\\
        $\blacksquare \blacksquare \blacksquare$\\
        $\square \square \square$\\
        $\blacksquare \blacksquare \blacksquare$\\
        GENERACION:1   POBLACION:3\\
        $\blacksquare \square \blacksquare$\\
        $\blacksquare \square \blacksquare$\\
        $\blacksquare \square \blacksquare$\\
        GENERACION:2   POBLACION:3\\
        $\blacksquare \blacksquare \blacksquare$\\
        $\square \square \square$\\
        $\blacksquare \blacksquare \blacksquare$\\
	    \hline
    \end{tabular}
\end{tabular}
\end{center}
