\section{Multiplicación Rusa}

Los Rusos tienen una forma única de multiplicar. Para ellos multiplicar consiste en hacer una tabla a partir del par de números a multiplicar, suponga que usted desea multiplicar $n_1 \cdot n_2$. La primera fila de la tabla seria $f_{0,0} = n_1, f_{0,1} = n_2$, la siguiente seria $f_{1,0}$ como la parte entera de  $f_{0,0}/2$, $f_{1,1} = 2 \cdot f_{0,1}$, luego $f_{2,0}$ como la parte entera de  $f_{1,0}/2$, $f_{2,1} = 2 \cdot f_{1,1}$, y así sucesivamente, mas formalmente:

$$f_{i,j} = \left\{ \begin{array}{lcc}
             \lfloor f_{i-1,j}/2 \rfloor &   si  & j=0 \\
             \\ 2 \cdot f_{i-1,j} &  si & j=1 \\
             \end{array}
   \right.$$
   
La tabla termina con $f_{i,0} = 1$.
La multiplicación se obtiene como la suma de todos los $f_{i,1}$, con $f_{i,0}$ impar.

\begin{center}
    Ejemplo $12 \cdot 12$\\
    \begin{tabular}{|l|l|}
    \hline
    $f_{0,0} = 12$               & $f_{0,1} = 12$                \\ \hline
    $f_{1,0} = 6$                & $f_{1,1} = 24$                \\ \hline
    $f_{2,0} = 3$             & $f_{2,1} = 48$             \\ \hline
    $f_{3,0} = 1$             & $f_{3,1} = 96$             \\ \hline
    \multicolumn{2}{|l|}{$12 \cdot 12 = f_{2,1} + f_{3,1} =144$} \\ \hline
    \end{tabular}
    
\end{center}

El problema que los Rusos le piden resolver es resolver la multiplicaciones del archivo \texttt{mult.txt} y registrar los resultados con las tablas en el archivo \texttt{mult\_sol.txt}, a continuación se detalla la estructura de los archivos:

\begin{center}
    \texttt{mult.txt} \\
    \begin{tabular}{|l|}
        \hline
        121x398\\
        432x993\\
        12x12\\
        1920x1080\\
        1020x720\\
        \hline
    \end{tabular}
\end{center}

Para poder resolver este problema le piden hacer las siguientes funciones:

\begin{itemize}
    \item[a.] Desarrollar la función \texttt{gen\_tabla(producto)} que reciba como parámetro una tupla con el par de números que se desean multiplicar y retorne una lista de tuplas con cada tupla representado una fila de la tabla:
    
    \begin{lstlisting}[style=consola]
>>> [*gen_tabla((12,12))*]
[(12,12),(6,24),(3,48),(1,96)]
    \end{lstlisting}

    \item[b.] Desarrollar la función \texttt{sumar\_impares(tabla)} que reciba como parámetro la tabla como una lista de tuplas, y retorne la suma de todos los $f_{i,1}$, con $f_{i,0}$ impar.
    \begin{lstlisting}[style=consola]
>>> [*sumar_impares(gen_tabla(12,12))*]
144
    \end{lstlisting}

    \item[c.] Programar la función \texttt{resolver(nombre\_arch)} que reciba como parámetro el nombre del archivo con las con las multiplicaciones a resolver y genere el archivo cuyo nombre se \texttt{nombre\_arch} concatenado con '\_sol.txt' con la solución del problema.
      \begin{lstlisting}[style=consola]
>>> [*resolver("mult.txt")*]
>>> 
    \end{lstlisting}
\end{itemize}

    % \newpage
    \begin{center}
            \texttt{mult\_sol.txt}\\
        	\begin{tabular}{|l|}
        		\hline
TABLA de 121x398=48158\\
1	121		398\\
0	60		796\\
0	30		1592\\
1	15		3184\\
1	7		6368\\
1	3		12736\\
1	1		25472\\
TABLA de 432x993=428976\\
0	432		993\\
0	216		1986\\
0	108		3972\\
0	54		7944\\
1	27		15888\\
1	13		31776\\
0	6		63552\\
1	3		127104\\
1	1		254208\\
        		\hline
        	\end{tabular}
        \end{center}