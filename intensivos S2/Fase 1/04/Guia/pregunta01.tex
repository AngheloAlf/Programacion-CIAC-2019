\section{Análisis de Palabras}

Programe las siguientes funciones
\begin{itemize}

    \item \texttt{contar\_letras(palabra)} que reciba una palabra en string y retorne un diccionario donde las llaves sean letras y los valores la cantidad de veces que está dicha letra en la palabra. Note que no diferencia letras mayúsculas de minúsculas.
    
    \begin{lstlisting}[style=consola]
>>> [*print(contar_letras('Ciac'))*]
{'i': 1, 'a': 1, 'c': 2}
>>> [*print(contar_letras('Intensivo'))*]
{'i': 2, 'n': 2, 't': 1, 'e': 1, 's': 1, 'v': 1, 'o': 1}
    \end{lstlisting}
    
    \item La función \texttt{son\_anagramas(p1,p2)} que retorne True o False si dos palabras \texttt{p1} y \texttt{p2} son anagramas. Dos palabras son anagramas si tienen las mismas letras pero en otro orden. 
    
    \begin{lstlisting}[style=consola]
>>> [*print(son_anagramas('grite','tigre'))*]
True
>>> [*print(son_anagramas('ciac','progra'))*]
False
    \end{lstlisting}
    
    \item Escriba la función \texttt{es\_panvocalica(palabra)} que retorne un booleano indicando si la palabra es panvocálico o no. Las palabras panvocálicas son aquellas que tienen las cinco vocales en ella.
    
    \begin{lstlisting}[style=consola]
>>> [*print(es_panvocalica('neumatico'))*]
True
>>> [*print(es_panvocalica('panvocalica'))*]
False
    \end{lstlisting}
    
    \item \texttt{en\_orden(palabra)} si todas las letras de \texttt{palabra} están en orden alfabético
    \begin{lstlisting}[style=consola]
>>> [*print(en_orden('himnos'))*]
True
>>> [*print(en_orden('zapato'))*]
False
    \end{lstlisting}
    
    \item \texttt{en\_orden\_segun(palabra,guia)} si todas las letras de \texttt{guia} van apareciendo en el mismo orden en \texttt{palabra}
    \begin{lstlisting}[style=consola]
>>> [*print(en_orden_segun('intensivo','iso'))*]
True
>>> [*print(en_orden_segun('limitacion','amc'))*]
False
    \end{lstlisting}
    
    \item \texttt{palabras\_repetidas(oracion)} que retorne una lista con las palabras que están repetidas en la oración.
    
    \begin{lstlisting}[style=consola]
>>> [*print(palabras_repetidas('El sobre esta sobre el escritorio'))*]
['el', 'sobre']
>>> [*print(palabras_repetidas('este intensivo es entretenido'))*]
[]
    \end{lstlisting}
\end{itemize}
