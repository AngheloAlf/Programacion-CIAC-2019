\section{Resumen de aprobados}

A la universidad Santa María, líder en ingeniería, magia y hechicería, llegan alumnos de intercambio provenientes de Hogwarts para tomar el curso de defensa contra las artes oscuras. Los profesores (magos) son nulos en el manejo de datos y le piden a los buenos estudiantes de IWI-131 (muggles) que van a los intensivos de CIAC ayuda con el filtro de alumnos aprobados y reprobados. Las reglas para aprobar son:
\begin{itemize}
    \item Que el promedio de sus notas sea mayor o igual a 55.
    \item Que el porcentaje de asistencia sea mayor al 75\%
\end{itemize}

De cumplirse ambas condiciones, el estudiante aprueba, de cumplirse sólo una, el estudiante debe dar un examen global y de no cumplirse ninguna el estudiante reprueba.

Se tienen los archivos: \texttt{notas.txt} y \texttt{asistencia.txt}:

\begin{center}
	\begin{tabular}{c c}
        \texttt{notas.txt} & \texttt{asistencia.txt}\\
    	\begin{tabular}{|l|}
    		\hline
Harry\#80:50:71:100\\
Neville\#30:25:60:100\\
Hermione\#100:100:99:100\\
Ronald\#55:50:45:100\\
Tom\#0:10:0:15\\
    		\hline
    	\end{tabular} & \begin{tabular}{|l|}
    		\hline
Hermione\#1:1:1:1:1:1:1\\
Ronald\#0:0:1:1:1:1:1\\
Tom\#0:0:0:1:0:1:1\\
Neville\#1:1:1:1:1:1:0\\
Harry\#1:1:1:1:0:1:1\\
    		\hline
    	\end{tabular}
	\end{tabular}
\end{center}

El archivo \texttt{notas.txt} tiene la estructura \texttt{nombre\_alumno\#nota1: ... :nota4}, mientras que el archivo \texttt{asistencia.txt} tiene la estructura \texttt{nombre\_alumno\#asis1: ... :asis7}. A usted se le pide:

\begin{itemize}
    \item[a.] Desarrollar la función \texttt{listado\_alumnos(nombre\_archivo)} que reciba como parámetro el nombre de uno de los dos archivos y retorne una lista con todos los alumnos.
    \begin{lstlisting}[style=consola]
>>> [*listado_alumnos('notas.txt')*]
['Harry', 'Neville', 'Hermione', 'Ronald', 'Tom']
    \end{lstlisting}

    \item[b.] Desarrollar la función \texttt{aprueba\_por\_notas(nombre\_archivo)} que reciba como parámetro el nombre del archivo con las notas y retorne un diccionario que asocie el nombre del alumno con un booleano según haya o no cumplido con el requisito número 1.
    \begin{lstlisting}[style=consola]
>>> [*aprueba_por_notas('notas.txt')*]
{'Harry':True, 'Neville':False, 'Hermione':True, 'Ronald':True, 'Tom':False}
    \end{lstlisting}

    \item[c.] Programar la función \texttt{aprueba\_por\_asistencia(nombre\_archivo)} que reciba como parámetro el nombre del archivo con las asistencias y retorne un diccionario que asocie el nombre del alumno con un booleano según haya o no cumplido con el requisito número 2.
    \begin{lstlisting}[style=consola]
>>> [*aprueba_por_asistencia('asistencia.txt')*]
{'Hermione':True, 'Ronald':False, 'Tom':False, 'Neville':True, 'Harry':True}
    \end{lstlisting}

    \item[d.] Crear la función \texttt{resumen(archivo\_notas, archivo\_asistencia)} que reciba los nombres de ambos archivos, y cree el archivo \texttt{Final.txt} con la estructura \texttt{nombre\#estado}, donde estado puede ser (APROBADO, EXAMEN GLOBAL, REPROBADO) de cada alumno. La función retorna None.

\end{itemize}

\begin{center}
    \texttt{final.txt} \\
	\begin{tabular}{|l|}
		\hline
Harry\#APROBADO\\
Neville\#EXAMEN GLOBAL\\
Hermione\#APROBADO\\
Ronald\#EXAMEN GLOBAL\\
Tom\#REPROBADO\\
		\hline
	\end{tabular}
\end{center}
