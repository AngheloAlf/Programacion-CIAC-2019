\section{La napydad}

Santa Claus ha comprado un sistema hecho en Python para administrar la Navidad de este año.

Básicamente, el sistema maneja un archivo, el cual primero tiene una linea que dice "\textit{Lista de buenos y malos 20XX:}", luego una linea en blanco y luego muchas lineas con el formato \texttt{nombre;x,y;tipo}, donde \texttt{x} e \texttt{y} son las coordenadas en el mapa de la ubicación de la persona y \texttt{tipo} indica como se ha portado la persona ('\texttt{B}': Bueno; '\texttt{M}': Malo).

Ver ejemplo:
\begin{center}
\texttt{personas.txt}\\
	\begin{tabular}{|l|}
		\hline
Erick Lopez;12.9887,1.5567;B\\
Diego;-2.9001,-13.2353;M\\
Cesar Moltedo;12.1986,2.5321;M\\
Miguel;13.1301,2.5463;B\\
Andrew Ng;-2.9001,7.6453;B\\
Gabriel;9.6201,-4.6673;M\\
Anghelo;12.2316,15.0089;M\\
El Barto;28.1326,57.1231;M\\
		\hline
	\end{tabular}
\end{center}

Tenga en consideración que esto es un ejemplo, el archivo real sera mas grande. Además las personas aparecen una única vez.

Ahora usted debe:
\begin{enumerate}

\item Desarrollar la función \texttt{persona\_mas\_cerca(personas, posicion)} que recibe un diccionario de personas y la posición actual de Santa Claus. La función debe retornar el nombre de la persona que sea buena mas cercana a la posición de Santa Claus. 

El diccionario \texttt{personas} tiene como llave el nombre de un persona, y como valor otro diccionario, el cual tiene como datos el par \texttt{'ubicacion': (x, y)}, que indica la ubicación de la persona y además del par \texttt{'tipo': 'B'}, el cual indica si es malo o bueno.
    
Para calcular esta distancia puede usar la formula $\sqrt{(x_2 - x_1)^2 + (y_2 - y_1)^2 }$.
    
\begin{lstlisting}[style=consola]
>>> [*personas = {"Erick Lopez": {"ubicacion":(12.9887,1.5567), "tipo": "B"}, 
"Andrew Ng": {"ubicacion": (-2.9001,7.6453), "tipo": "M"}}*]
>>> [*print(persona_mas_cerca(personas, (12.9676, 1.4991)))*]
'Erick Lopez'
\end{lstlisting}

\item Santa Claus ha determinado que la mejor ruta para repartir todos los regalos es ir siempre a la persona mas cercana (con cuidado de no repetirlas) desde la posición en la que se encuentra. Usted debe crear la función \texttt{mejor\_ruta(archivo, posicion)} que reciba el nombre del archivo y la posición inicial de Santa Claus, y retorne una lista con la ruta a seguir identificada por el nombre
de las personas buenas.

% HINT: Se recomienda usar la funcion anterior.

\begin{lstlisting}[style=consola]
>>> [*print(mejor_ruta("personas.txt", (0.0, 0.0)))*]
['Andrew Ng', 'Miguel', 'Erick Lopez']
\end{lstlisting}

\item A pocos días de la Navidad, Santa Claus se ha dado cuenta que la lista con personas posee
errores. Santa ha concluido que el Grinch había hackeado el sistema alterando el campo \texttt{tipo} de
algunas personas, es decir, una persona buena fue cambiada a mala o vise-versa. Por suerte Santa
tiene un respaldo, una lista con tuplas con el nombre y el tipo de cada persona. 

Ver ejemplo:

\begin{lstlisting}[style=consola]
respaldo = [('Erick Lopez', 'B'), ('Erick Lopez', 'B'), ('Diego', 'B'), 
('Cesar Moltedo', 'M'), ('Miguel', 'B'), ('Andrew Ng', 'M'), 
('Gabriel', 'B'), ('Anghelo', 'M'),('El Barto', 'M')]
\end{lstlisting}

Ahora usted debe implementar la función \texttt{restaurar(archivo, respaldo)} que reciba
como parámetro los archivos de personas y respaldo. La función debe restaurar el archivo de las personas,
es decir, volver el tipo de cada persona (según respaldo). La función retorna \texttt{None}.

\begin{lstlisting}[style=consola]
>>> [*restaurar("personas.txt", respaldo)*]
>>> 
\end{lstlisting}

\begin{center}
\texttt{personas.txt}\\
	\begin{tabular}{|l|}
		\hline
Erick Lopez;12.9887,1.5567;B\\
Diego;-2.9001,-13.2353;B\\
Cesar Moltedo;12.1986,2.5321;M\\
Miguel;13.1301,2.5463;B\\
Andrew Ng;-2.9001,7.6453;M\\
Gabriel;9.6201,-4.6673;B\\
Anghelo;12.2316,15.0089;M\\
El Barto;28.1326,57.1231;M\\
		\hline
	\end{tabular}
\end{center}

\end{enumerate}
