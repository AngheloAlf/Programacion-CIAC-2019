\section{Polinomios}

Un polinomio de grado \(n\)
es una función matemática de la forma
\begin{equation*}
p(x) = a_0
  + a_1 \cdot x
  + a_2 \cdot x^2
  + \hdots
  + a_n \cdot x^n
\end{equation*}
Los valores \(a_0, \, a_1, \, \hdots \, , \, a_n\)
son los coeficientes del polinomio,
y \(x\) es la variable independiente.
Desarrolle un programa que evalúe un polinomio.

Primero, el usuario debe ingresar el valor de $x$ que va a ser evaluado. A continuación, luego el usuario ingresara los coeficientes en orden, empezando por el coeficiente $a_0$, $a_1$, ... hasta $a_n$.

Para indicar que todos los coeficientes han sido ingresados, se debe escribir el texto \texttt{FIN}. 

Finalmente, el programa debe mostrar el valor calculado de $p(x)$. El ejemplo muestra como evaluar el polinomio $p(x) = -7 - 3x^2 + 2.4x^3$, considerando $x = 2.1$.

\begin{lstlisting}[style=consola]
x: [*2.1*]
Coeficientes:
a0: [*-7*]
a1: [*0*]
a2: [*-3*]
a3: [*2.4*]
a4: [*FIN*]
p(2.1) = 1.9964000000000013
\end{lstlisting}
