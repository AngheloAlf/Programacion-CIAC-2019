\section{Concurso de clavados}
Piscina USM está organizando un concurso de clavados, para lo cual necesita la ayuda de los estudiantes que van a los intensivos de programación en CIAC para desarrollar un programa que calcule los puntajes de los concursantes.

El programa debe preguntar cuantos puntajes se ingresarán, para después indicar cual fue el puntaje más alto, el más bajo y entregar el puntaje final que es el promedio de los números ingresados \textit{sin considerar ni el más alto ni el más bajo}.

Los puntajes irán en un rango de 0 a 10. Además su programa debe realizar un feedback hacia el concursante dependiendo de su puntaje final.
\begin{itemize}
    \item Si el puntaje va desde 8 a 10 responder "\texttt{Excelente!}"
    \item Si el puntaje va desde 5 a 8 responder "\texttt{Muy bien!}"
    \item Si el puntaje va desde 3 a 5 responder "\texttt{No esta mal}"
    \item Si no se cumple ninguna anterior, debe responder "\texttt{Esto no es un concurso de guatazos, sigue practicando!}"
\end{itemize}

% \begin{lstlisting}[style=consola]
% Ingrese numero de puntajes: [*7*]
% Puntaje 1: [*7.7*]
% Puntaje 2: [*3.4*]
% Puntaje 3: [*5.7*]
% Puntaje 4: [*8.9*]
% Puntaje 5: [*6.5*]
% Puntaje 6: [*5.6*]
% Puntaje 7: [*4.2*]
% El puntaje minimo fue 3.4
% El puntaje maximo fue 8.9
% El puntaje final es 5.94
% Muy bien!
% \end{lstlisting}

% En el ejemplo se ve como los puntajes \texttt{8.9} y \texttt{3.4} fueron omitidos en el cálculo del promedio.

\begin{lstlisting}[style=consola]
Ingrese numero de puntajes: [*4*]
Puntaje 1: [*3.4*]
Puntaje 2: [*2.2*]
Puntaje 3: [*1.0*]
Puntaje 4: [*7.8*]
El puntaje minimo fue 1.0
El puntaje maximo fue 7.8
El puntaje final es 2.8
Esto no es un concurso de guatazos, sigue practicando!
\end{lstlisting}

En el ejemplo se ve como los puntajes \texttt{1.0} y \texttt{7.8} fueron omitidos en el cálculo del promedio.