\section{¡Funciones!}

Una función es un conjunto de instrucciones, las cuales tienen con objetivo ser re-utizadas de forma fácil y bonita.

Las funciones suelen trabajar con datos de entrada entregados por el usuario, hace cálculos mágicos y místicos, y entrega un valor de salida.

Por ejemplo, piensa en la función \texttt{sqrt(x)}. Esta función recibe un número positivo como parámetro, y \textit{retorna} la raíz cuadrada de dicho número.

Las funciones no tienen porque recibir un único parámetro, estas pueden ser la cantidad que sea, por ejemplo  la función \texttt{max(x, y)} recibe 2 parámetros, y \textit{retorna} el mayor entre esos dos.

¡Mas adelante también veremos funciones que retornan nada y otras muchas características entretenidas de las funciones!

Dicho todo esto, y a modo de introducción para que empiecen a \textit{soltar la mano} con las funciones, deberán implementar las siguientes funciones:

\begin{enumerate}
    \item La ecuación matemática $f(x) = -x^{2} - 3x + 12$.
    \begin{lstlisting}[style=consola]
>>> [*f(5)*]
-28
>>> [*f(0)*]
12
    \end{lstlisting}
    
    \item Entretenido, ¿no? Generalicemos la recta anterior para valores cualquiera, implementando la función $f\_gen(x, a, b, c) = a*x^2 + b*x + c$.
    \begin{lstlisting}[style=consola]
>>> [*f_gen(5, 2, 4, 0.5)*]
70.5
>>> [*f_gen(0, -1, -3, 12)*]
12
    \end{lstlisting}

    \item ¿Muy fácil para ti? Creo que te he sub-estimado. Enfréntate a la siguiente sumatoria:
    \begin{equation*}
        \text{sumator(n, a, b, c)} = \frac{1}{n}\sum_{i=0}^{n}{f\_gen(i, a, b, c)}
    \end{equation*}
    \begin{lstlisting}[style=consola]
>>> [*sumator(7, 2, -3, 12)*]
29.0
>>> [*sumator(5, 0, 9, -3)*]
15.0
    \end{lstlisting}
    
    \item ¿Has derrotado a ese esbirro? Veamos que haces contra \textit{Heaviside} $._{Muajajajaja}$:
    \begin{equation*}
        \text{heav(n)} = \frac{1}{n}\sum_{i=0}^{n}\frac{f(i)*i^2}{n}*H\left(\frac{(-1)^i * f(i)*i^2}{n}, i\right)
    \end{equation*}
    
    Donde la función $H(x, n)$ esta definida como:
    \begin{equation*}
        H(x, n) = 
        \begin{cases}
        1 & \quad\text{si } x \geq n \\
        0 & \quad\text{si } x < n \\
     \end{cases}
    \end{equation*}

    \begin{lstlisting}[style=consola]
>>> [*heav(7)*]
-16
>>> [*heav(13)*]
-170
    \end{lstlisting}
    
        \begin{comment}
    
    \item Wow, pudiste con eso. Bueno, a que no te esperas esto: 
    
    Implementa una función \texttt{palabra\_de\_numeros(pal)}, la cual recibe un \texttt{string} que contiene únicamente números. El valor retornado por esta función es la suma de todos esos números, exceptuando el numero mayor y el numero menor. Considere que la palabra entregada tiene al menos 2 digitos.
    
    \begin{lstlisting}[style=consola]
>>> [*palabra_de_numeros("123456")*]
14
>>> [*palabra_de_numeros("987654")*]
26
    \end{lstlisting}
    
        \end{comment}
\end{enumerate}

Veo que has logrado derrotarme, me retirare por ahora...

