\section{Impuestos huyentes}

En el glorioso reino de \textit{Pythonia} las empresas se categorizan según el nivel de utilidades que tengan. Si sus utilidades son menores a \$50 millones se consideran una empresa pequeña, mediana si recauda entre \$50 y \$80 millones, y grande si recauda mas de \$80 millones.

Nuestro glorioso reino tiene como ley que las grandes empresas deben pagar un 40\% de sus utilidades como impuestos, un 30\% si son medianas empresas y un 25\% las empresas pequeñas. A pesar de esto, se ha detectado que algunas empresas evaden el pago de impuestos gracias a vacíos legales. Después de varios estudios se ha determinado que:

\begin{itemize}
    \item Todas las empresas grandes que pagan mas de \$64 millones en impuestos se van al extranjero y no pagan impuestos a Pythonia. En caso contrario, si paga mas de \$36 millones esta se divide entre 2 empresas, cada una repartiéndose las utilidades por igual y pagando los impuestos como corresponde. En cualquier otro caso se pagan los impuestos debidos.
    \item Las empresas medianas que ganan sobre \$70 millones contratan un buen contador que les hace pagar un 30\% menos de impuestos de lo que corresponde. Si ganan sobre \$60 millones, contratan a un contador de bajo presupuesto que les hace pagar un 10\% menos. El resto paga la cantidad real.
    \item Las pequeñas empresas siempre pagan lo que les corresponde. Pero las que ganan menos de \$20 millones no tienen recursos para contratar un contador, por lo que siempre se equivocan y terminan pagando un 20\% extra de impuestos por multa.
\end{itemize}

En base a lo anterior, se le pide lo siguiente:

\begin{enumerate}
    \item[$\alpha$] Escriba la función \texttt{impuestos\_por\_ley(utilidad)}; la cual recibe un número que indica la utilidad de la empresa, y retorna lo que dicha empresa debería pagar según la ley.
    
    \begin{lstlisting}[style=consola]    
>>> [*print(impuestos_por_ley(140))*]
56.0
>>> [*print(impuestos_por_ley(68))*]
20.4
>>> [*print(impuestos_por_ley(12))*]
3.0
    \end{lstlisting}
    
    \item[$\beta$] Cree la función \texttt{empresa\_grande(utilidad)}, la cual recibe un numero indicando la utilidad que recibe una empresa grande y retorna lo que pagaría al intentar evadir impuestos.
    
    \begin{lstlisting}[style=consola]
>>> [*print(empresa_grande(140))*]
42.0
>>> [*print(empresa_grande(95))*]
23.5
    \end{lstlisting}
    
    \item[$\gamma$] Programe la función \texttt{empresa\_mediana(utilidad)}, la cual recibe un numero indicando la utilidad que recibe una empresa mediana y retorna lo que pagaría al intentar evadir impuestos.
    
    \begin{lstlisting}[style=consola]
>>> [*print(empresa_mediana(75))*]
15.75
>>> [*print(empresa_mediana(68))*]
18.36
    \end{lstlisting}
    \newpage
    \item[$\theta$] Programe la función \texttt{empresa\_pequena(utilidad)}, la cual recibe un numero indicando la utilidad que recibe una empresa pequeña y retorna lo que pagaría al intentar pagar sus impuestos.
    \begin{lstlisting}[style=consola]
>>> [*print(empresa_pequena(35))*]
8.75
>>> [*print(empresa_pequena(12))*]
3.6
    \end{lstlisting}
    
    \item[$\star$] Escriba un \textbf{programa} que reciba los ingresos de distintas empresas y que sume el impuesto que están pagando y el que deberían pagar por ley. Además, el programa debe poder decir cual fue la empresa que pago menos impuestos y la empresa que pago mas. Este programa debe pedir que se ingresen utilidades de empresas hasta que se ingrese el numero 0.
    
    Considere que las cantidades ingresadas están en millones.
    

    \begin{lstlisting}[style=consola]
Ingrese utilidad empresa 1: [*200*]
Ingrese utilidad empresa 2: [*18*]
Ingrese utilidad empresa 3: [*95*]
Ingrese utilidad empresa 4: [*55*]
Ingrese utilidad empresa 5: [*0*]
El impuesto esperado es 139.0 
pero lo pagado fue 45.65
La empresa que pago menos impuestos fue la empresa 1 
la cual pago 0
La empresa que pago mas impuestos fue la empresa 3 
la cual pago 23.75
    \end{lstlisting}

    \begin{lstlisting}[style=consola]
Ingrese utilidad empresa 1: [*8*]
Ingrese utilidad empresa 2: [*74*]
Ingrese utilidad empresa 3: [*68*]
Ingrese utilidad empresa 4: [*180*]
Ingrese utilidad empresa 5: [*55*]
Ingrese utilidad empresa 6: [*92*]
Ingrese utilidad empresa 7: [*0*]
El impuesto esperado es 169.9 pero lo pagado fue 75.8
La empresa que pago menos impuestos fue la empresa 4 la cual pago 0
La empresa que pago mas impuestos fue la empresa 6 la cual pago 23.0

    \end{lstlisting}
    
\end{enumerate}
